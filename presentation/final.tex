\documentclass[11pt]{beamer}
\usetheme[pageofpages=of,% String used between the current page and the
                         % total page count.
          bullet=circle,% Use circles instead of squares for bullets.
          titleline=true,% Show a line below the frame title.
          alternativetitlepage=true,% Use the fancy title page.
          ]{Torino}

\usepackage{color}
\usepackage[utf8]{inputenc}
\usepackage[T1]{fontenc}
\usepackage[ngerman,english]{babel}
\usepackage{url}
\usepackage{tangocolors}
\usepackage{tikz}
\usetikzlibrary{arrows,backgrounds,snakes,mindmap}

\setbeamerfont{subtitle}{size=\Large}
\setbeamerfont{author}{size=\large}
\setbeamerfont{institute}{size=\large}
\graphicspath{{img/}}

\author{Daniel Siegel}
\title{On the New Threats of Social Engineering}% exploiting social networks}
\subtitle{Exploiting Social Networks}
\institute{13. August 2009}
\date{}

\begin{document}
\begin{frame}[t,plain]
\titlepage
\end{frame}

\begin{frame}[t]{Inhalt}
  \begin{itemize}
    \item Motivation \& Problemstellung
    \item Social Engineering
    \item Soziale Netzwerke
    \item Ein konkretes Soziales Netzwerk: Twitter
    \item Prototyp
    \item Evaluation \& Ausblick
  \end{itemize}
\end{frame}

\begin{frame}[t]{Ausgangssituation}
  \begin{center}
    \begin{tikzpicture}[scale=3]

      \fill[fill=chameleon2] (0,0) -- (0:1) arc (0:90:1) -- cycle
      node[white,font=\Huge] at (45:0.5) {1};
      \draw (45:1) -- (45:1.3) node[right] {Information};

      \fill[fill=skyblue1!40] (0,0) -- (270:1) arc (270:360:1) -- cycle
      node[white,font=\Huge] at (315:0.5) {2};
      \draw [black!40] (315:1) -- (315:1.3) node[right, black!40] {Vertrauen};

      \fill[fill=scarletred1!40] (0,0) -- (180:1) arc (180:270:1) -- cycle
      node[white,font=\Huge] at (225:0.5) {3};
      \draw [black!40] (225:1) -- (225:1.3) node[left, black!40] {Ausbeutung};

      \fill[fill=butter2!40] (0,0) -- (90:1) arc (90:180:1) -- cycle
      node[white,font=\Huge] at (135:0.5) {4};
      \draw [black!40] (135:1) -- (135:1.3) node[left, black!40] {Angriff};

    \end{tikzpicture}

  \end{center}
\end{frame}

\begin{frame}[t]{Problemstellung}

\vskip3em
\begin{beamercolorbox}[sep=1em]{title page header}
Wie können Daten bestimmter
Individuen bzw. Unternehmen über Soziale Plattformen automatisch extrahiert und
so dargestellt werden, dass sie für einen Social Engineering Angriff benutzt
werden können. Zudem sollen Gegenmaßnahmen zu diesen Angriffen erarbeitet
werden.
\end{beamercolorbox}
\end{frame}

\begin{frame}[t]{Definition: Social Engineering}

\vskip3em
\begin{beamercolorbox}[sep=1em]{title page header}

Social Engineering uses influence and persuasion to deceive people by
convincing them that the social engineer is someone\\ he is not, or by
manipulation. As a result, the social engineer is able to take advantage of
people to obtain information with\\ or without the use of technology.

\flushright{\scriptsize Kevin D. Mitnick and William L. Simon. \glqq{}The Art of Deception\grqq{}\hskip1em}
\end{beamercolorbox}
\end{frame}

\begin{frame}[t]{Definition: Social Engineering}

\vskip3em
\begin{beamercolorbox}[sep=1em]{title page header}

Social Engineering uses influence and persuasion to
\textbf{\textcolor{skyblue3}{deceive}} people by
\textbf{\textcolor{skyblue3}{convincing}} them
that the social engineer is someone he is not, or by manipulation. As a result,
the social engineer is able to \textbf{\textcolor{skyblue3}{take advantage}} of people
to \textbf{\textcolor{skyblue3}{obtain information}} with or without the use of
technology.

\flushright{\scriptsize Kevin D. Mitnick and William L. Simon. \glqq{}The Art of Deception\grqq{}\hskip1em}
\end{beamercolorbox}
\end{frame}

\begin{frame}[t]{Verlaufszyklus eines Angriffes}
  \begin{center}
    \begin{tikzpicture}[scale=3]

      \fill[fill=chameleon2] (0,0) -- (0:1) arc (0:90:1) -- cycle
      node[white,font=\Huge] at (45:0.5) {1};
      \draw (45:1) -- (45:1.3) node[right] {Information};

      \fill[fill=skyblue1] (0,0) -- (270:1) arc (270:360:1) -- cycle
      node[white,font=\Huge] at (315:0.5) {2};
      \draw (315:1) -- (315:1.3) node[right] {Vertrauen};

      \fill[fill=scarletred1] (0,0) -- (180:1) arc (180:270:1) -- cycle
      node[white,font=\Huge] at (225:0.5) {3};
      \draw (225:1) -- (225:1.3) node[left] {Ausbeutung};

      \fill[fill=butter2] (0,0) -- (90:1) arc (90:180:1) -- cycle
      node[white,font=\Huge] at (135:0.5) {4};
      \draw (135:1) -- (135:1.3) node[left] {Angriff};

    \end{tikzpicture}

  \end{center}
\end{frame}

\begin{frame}{Ziele}
  \begin{itemize}
    \item Finanzielle Bereicherung
    \item Industriespionage
    \item Spass - Macht
    \item Identitätsdiebstahl
    \item Datendiebstahl
    \item Soziale Überlegenheit
  \end{itemize}
\end{frame}

\begin{frame}{Informationen sammeln}
  \begin{itemize}
    \item Möglichkeiten der Informationsgewinnung
      \begin{itemize}
      \item Firmen-Website
      \item Persönliche Homepage
      \item Dumpster Diving
      \item Phishing
      \item Trojanische Pferde
      \item Newsgroups \& Mailinglisten
      \item Job Sites
      \item \dots
      \end{itemize}
    \item Ziel: Informationen $\Rightarrow$ Vertrauen
  \end{itemize}
\end{frame}

\begin{frame}{Vertrauen erarbeiten}
  \begin{itemize}
    \item Informationen
    \item Verantwortung der Handlungen des Opfers entnehmen
    \item Hilfestellung
    \item Beziehung
    \item Stellung
    \item \dots
  \end{itemize}
\end{frame}

\begin{frame}[t]{Ausbeutung, Manipulation \& Angriff}
  \begin{columns}
  \column{.50\textwidth}
  \vspace{3.8em}
  \begin{itemize}
    \item Human Based Attacks
    \begin{itemize}
      \item Identitätsdiebstahl
      \item Vortäuschen von Ermächtigungen
      \item Technischer Support
      \item Reverse Social Engineering
      \item Shoulder-Surfing
      \item Dumpster-Diving
      \item Persönlicher Auftritt
    \end{itemize}
  \end{itemize}
  \column{.50\textwidth}
  \pause
  \begin{itemize}
    \item Computer Based Attacks
    \begin{itemize}
      \item Phishing
      \item Spam
      \item Malware
      \item Suggestion vertrauenswürdiger Quelle
    \end{itemize}
  \end{itemize}
  \end{columns}
\end{frame}

\begin{frame}[t]{Soziale Netzwerke}
  \begin{center}
    \includegraphics[width=0.75\textwidth]{fbmap}

    {\tiny
    http://venturebeat.com/2009/04/08/trying-to-analyze-facebooks-latest-statistics-more-status-updates-more-content-sharing/
    }
  \end{center}
\end{frame}

\begin{frame}{Soziale Netzwerke - Angriffe}
  \begin{itemize}
    \item \textit{Passive} Attacken
    \item \textit{Aktive} Attacken
  \end{itemize}
\end{frame}

\begin{frame}[t]{Ein konkretes Soziales Netzwerk: Twitter}
  \begin{center}
    \includegraphics[width=0.75\textwidth]{twitter_logo}
  \end{center}
  \begin{itemize}
    \item Soziales Netzwerk \& Microblogging Anbieter
    \item Gegründet 2006
    \item Mehr als $\sim$ 1.780.000 User
    \item Genutzt von Privatpersonen \& Unternehmen
    \item \texttt{http://www.twitter.com/\textit{[username]}}
  \end{itemize}
\end{frame}

\begin{frame}[t]
  \begin{center}
    \includegraphics[width=\textwidth]{twitter_home}
  \end{center}
\end{frame}

\begin{frame}[t]{Tweet}
\vskip1em
\begin{beamercolorbox}[sep=1em]{title page header}
\glqq{}@max you should check the weather 
  forecast, see http://tinyurl.com/plmr3m\grqq{}

{\scriptsize \textcolor{aluminium1}{9:34 AM May 28th from web in reply to max}}
\end{beamercolorbox}
\begin{itemize}
  \item 140 Zeichen
  \item @username, \#topic, \dots
  \item Uhrzeit
  \item Benutzte Infrastruktur
  \item reply, new message
  \item Kurz-URL
\end{itemize}
\end{frame}

\begin{frame}[t]
\vspace{-0.85em}
  \begin{center}
    \begin{tikzpicture}[scale=0.65, transform shape,
                         root concept/.append style={concept color=skyblue1},
                         level 1 concept/.append style={concept color=chameleon2},
                         text=white, mindmap]

    \tikzstyle{every annotation}=[fill=skyblue3, font=\sf]

    \node[concept] (Twitter) {Twitter}
      child[concept, grow=160] {node [concept] {Web Interface}}
      child[concept, grow=125] {node [concept] {Twitter API}
      child[concept, grow=left] {node [concept] {User Applications}}}
      child[concept, grow=90] {node [concept] {Facebook}}
      child[concept, grow=55] {node [concept] {IM}}
      child[concept, grow=20] {node [concept] {SMS}}
      %
      child[concept, grow=200] {node [concept] {Web Interface}}
      child[concept, grow=235] {node [concept] {Twitter API}
      child[concept, grow=170] {node [concept] {RSS}}
      child[concept, grow=210] {node [concept] {User Applications}}}
      child[concept, grow=270] {node [concept] {Facebook}}
      child[concept, grow=305] {node [concept] {IM}}
      child[concept, grow=340] {node [concept] {SMS}};

    \node[annotation] at (52.5:7.5) {Twitter Input Methods};
    \node[annotation] at (307.5:7.5) {Twitter Output Methods};

    \draw (2,0) -- (6,0);
    \draw (-2,0) -- (-6,0);

  \end{tikzpicture}
  \end{center}
\end{frame}

\begin{frame}[t]{Twitter API}
\begin{itemize}
  \item http://apiwiki.twitter.com/Twitter-API-Documentation
  \item GET, POST HTML-Methods
  \item GET Anfragen beschränkt auf 150 Requests pro Stunde
  \item JSON, XML, RSS, Atom
\end{itemize}
\vskip1em
\begin{beamercolorbox}[sep=1em]{title page header}
\textcolor{plum3}{GET}
http://twitter.com/\textcolor{skyblue3}{users/show}/\textcolor{scarletred2}{barackobama}.\textcolor{aluminium3}{json}
\end{beamercolorbox}
\end{frame}

\begin{frame}[t]{Twitter API (2)}
\begin{beamercolorbox}[sep=1em]{title page header}
{\scriptsize
\begin{tabular}{ll}
created\_at:   &    "Mon Mar 05 22:08:25 +0000 2007"\\
description: &     "The president of the United States of America"\\
favourites\_count: &  0\\
followers\_count:&  1221122\\
friends\_count: &   776706\\
Id:    &       813286\\
location: &        "Chicago, IL"\\
name:   &        "Barack Obama"\\
profile\_image\_url: & "http://s3.amazonaws.com/twitter\_production/ $\hookrightarrow$\\
             &   profile\_images/219314140/obama\_4color\_omark\_normal.jpg"\\
protected:   &   false\\
screen\_name: &   "BarackObama"\\
statuses\_count: &  272\\
time\_zone:  &    "Central Time (US \& Canada)"\\
url:       &   "http://www.barackobama.com"
\end{tabular}
}
\end{beamercolorbox}
\end{frame}

\begin{frame}[t]{Prototyp}
  \vspace{-0.85em}
  \begin{center}
    \begin{tikzpicture}[scale=0.52, transform shape,
                        root concept/.append style={concept color=skyblue1},
%                       level 1 concept/.append style={concept color=chameleon2},
                        text=white, mindmap]

      \tikzstyle{level 2 concept}+=[sibling angle=45]

      \path[mindmap]
        node[concept] {User}
        child[concept color=scarletred1, grow=0] {
          node[concept] {Friends}
          [clockwise from=90]
          child { node[concept] {Replies} }
          child { node[concept] {Shared Interests} }
          child { node[concept] {Friends} }
          child { node[concept] {Colleagues} }
          child { node[concept] {Followers} }
        }
        child[concept color=butter2, grow=90] {
          node[concept] at (90:1) {General data}
          [clockwise from=112.5]
          child { node[concept] {Name/ Username} }
          child { node[concept] {Personal description} }
          child { node[concept] {Homepage} }
          child { node[concept] {Picture} }
          child { node[concept] {Time Zone} }
          child { node[concept] {Location} }
          child { node[concept] {Profile creation date} }
          child { node[concept] {Messages} }
        }
        child[concept color=chameleon2, grow=180] {
          node[concept] {Messages} 
          [clockwise from=295]
          child { node[concept] {E-mail addresses} }
          child { node[concept] {Times} }
          child { node[concept] {Replies} }
          child { node[concept] {Input sources} }
          child { node[concept] {Interests} }
          child { node[concept] {Locations} }
        };
    \end{tikzpicture}
  \end{center}
\end{frame}

\begin{frame}
  \begin{center}
  {\Huge DEMO}\\
  \end{center}
\end{frame}

\begin{frame}{Evaluation}
  \begin{itemize}
    \item Informationsphase wird ersetzt bzw. erleichtert
    \item Geringeres Risiko entdeckt zu werden
    \item Firmen \& Privatpersonen verwundbar
    \item Ausblick
    \begin{itemize}
      \item Zusätzliche Quellen
      \item Genauere Informationsanalyse
      \item Automatisierte Angriffe
    \end{itemize}
  \end{itemize}
\end{frame}

\begin{frame}{Related Work / Social Engineering}
  \begin{itemize}
    \item The Art of Deception
    \begin{itemize}
      \item K. Mitnick, W. Simon
      \item ISBN 076454280X
    \end{itemize}
    \item Social Phishing
    \begin{itemize}
      \item T. Jagatic, N. Johnson, M. Jakobsson, F. Menczer
      \item http://doi.acm.org/10.1145/1290958.1290968
    \end{itemize}
    \item Social Engineering: The “Dark Art”
    \begin{itemize}
      \item T. Thornburgh
      \item http://doi.acm.org/10.1145/1059524.1059554
    \end{itemize}
  \end{itemize}
\end{frame}

\begin{frame}{Related Work / Soziale Netzwerke}
  \begin{itemize}
    \item Social Networks and Context-Aware Spam
    \begin{itemize}
      \item  G. Brown, T. Howe, M. Ihbe, A. Prakash, K. Borders
      \item http://doi.acm.org/10.1145/1460563.1460628
    \end{itemize}
    \item Information Revelation and Privacy in Online Social Networks
    \begin{itemize}
      \item R. Gross, A. Acquisti, H.  John
      \item http://doi.acm.org/10.1145/1102199.1102214
    \end{itemize}
  \end{itemize}
\end{frame}

\begin{frame}{Related Work / Twitter}
  \begin{itemize}
    \item A Few Chirps About Twitter
    \begin{itemize}
      \item B. Krishnamurthy , P. Gill, M. Arlitt
      \item http://doi.acm.org/10.1145/1397735.1397741
    \end{itemize}
    \item Why We Twitter
    \begin{itemize}
      \item A. Java, X. Song, T. Finin, B. Tseng
      \item http://doi.acm.org/10.1145/1348549.1348556
    \end{itemize}
  \end{itemize}
\end{frame}

\begin{frame}
  \begin{center}
  {\Huge Fragen?}\\

\vfill
  \texttt{git clone http://home.cs.tum.edu/siegel/dev/thesis.git}
  \end{center}
\end{frame}

\end{document}
