% Abstract for the TUM report document

\clearemptydoublepage
\phantomsection
\addcontentsline{toc}{chapter}{Abstract}

\selectlanguage{ngerman}

\vspace*{2cm}
\begin{center}
{\Large \bf Zusammenfassung}
\end{center}
\vspace{1cm}

Diese Arbeit behandelt eine Studie zu neuen Sicherheitslücken in Sozialen
Netzwerken, die durch Social Engineering ausgenutzt werden können. Soziale
Netzwerke sind weit verbreitet und werden oft für den privaten aber auch den
geschäftlichen Bereich benutzt. Da die Natur eines solchen Sozialen Netzwerkes
das Verteilen von Daten ist, stellt sich unweigerlich die Frage, ob diese
missbraucht werden können.

Social Engineering ist seit je her eine Bedrohung, die sowohl für Unternehmen
aber auch für Privatpersonen eine große Gefahr darstellt und nur sehr schwer zu
bekämpfen ist. Bislang war die Informationssuche, auf die ein Social
Engineering Angriff aufbaut aber nur mühsam zu bewerkstelligen und barg immer
das Risiko für den Angreifer, entdeckt zu werden.

Soziale Netzwerke werden nun in dieser Arbeit analysiert, inwiefern man
automatisch an Daten bestimmter Benutzer oder Unternehmen kommen kann um diese für
Social Engineering Angriffe zu benutzen. Dafür werden Social Engineering
Angriffe sowie Soziale Netzwerke analysiert und drei Beispielangriffe erstellt.
Diese werden dann von einem Prototypen implementiert und auf dem Sozialen
Netzwerk \textit{Twitter} durchgeführt. Anschließend werden diese Attacken
evaluiert und auf ihre Gefährlichkeit hin untersucht. Dabei stellt sich heraus,
dass die meisten Informationen, die ein Social Engineering Angriff benötigt,
sehr einfach und ohne Mitwissen des Opfers aus einem Sozialen Netzwerk
extrahiert werden kann. Neben einem Ausblick auf weitere mögliche Angriffe, werden
auch Schutzmöglichkeiten diskutiert.

\selectlanguage{english}
