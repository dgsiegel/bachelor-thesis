\chapter{Related Work}
\label{chap:relatedwork}

\section{Social Engineering}

In most companies and private networks, the main security focus aims towards
security technology, such as firewalls or anti-virus programs and other defense
strategies~\cite{winkler1995}. However, companies and private people are often
unaware of social engineering which sometimes is a lot more dangerous than the
IT-security suggests~\cite{jones2004}. Kevin Mitnick, a hacker and one of the
most famous social engineers, states that it is often easier to use social
engineering to get access to a system than searching for security
holes~\cite{mitnick2003}.

The term social engineering is not easy to describe and there are many
definitions available. 
The new hacker's dictionary~\cite{raymond1996} defines the term social engineering as
follows:
\begin{quote}
\textit{Term used among crackers and samurai for cracking techniques that rely on
weaknesses in wetware rather than software; the aim is to trick people into
revealing passwords or other information that compromises a target system's
security. [\dots]}
\end{quote}

Another way to define the term is that of Microsoft~\cite{microsoft2009}:
\begin{quote}
\textit{Social engineering is a way for criminals to gain access to your computer. The
purpose of social engineering is usually to secretly install spyware or other
malicious software or to trick you into handing over your passwords or other
sensitive financial or personal information.}
\end{quote}

A more general approach has been undertaken by Mitnick and Simon~\cite{mitnick2003}, which
is one of the most appropriate definitions and will be used in this work:
\begin{quote}
\textit{Social Engineering uses influence and persuasion to deceive people
by convincing them that the social engineer is someone he is not,
or by manipulation. As a result, the social engineer is able to take
advantage of people to obtain information with or without the use of
technology.}
\end{quote}


Besides the various definitions it is also important to understand how a
social engineer relates to a hacker in general. Thornburgh
proposes~\cite{thornburgh2004} that social engineers are not hackers
themselves, but are \textit{hacker-enablers}. Following this line of thought,
the goal of the social engineer is to gain either physical or digital direct
access to information of the target or an information system. Afterwards the
social engineer can enable a hacker to access and penetrate the system, delete,
alter or steal information and disrupt services~\cite{thornburgh2004}. Of
course, the social engineer can also be the hacker. Hacking characteristically
involves access to systems through technical means, while the social engineer
manipulates people to give him access to information that normally would not
be available~\cite{jones2004}. Furthermore, social engineering uses psychology
and theories about the human mindset, for instance what people expect from each
other and their natural tendency to be helpful~\cite{jones2004}.


\subsection{The Social Engineering Life Cycle}

Defining the profile of a social engineering attack is quite difficult, as
every attack includes people with their behavioural changes over time, just
like their mood and other personal or emotional characteristics. However,
Mitnick and Simon identified four stages of a social engineering
attack~\cite{mitnick2003}: research, developing rapport and trust, exploiting
trust and utilizing information. A single cycle does not have to be limited to
a singular cycle, but can contain several other cycles until the objective is
reached~\cite{thornburgh2004}. The process itself can therefore be recursive
and iterative, depending on the attack and methods used.

\begin{figure}
  \begin{center}
    \begin{tikzpicture}[scale=3]

      \fill[fill=chameleon2] (0,0) -- (0:1) arc (0:90:1) -- cycle
      node[white,font=\Huge] at (45:0.5) {1};
      \draw (45:1) -- (45:1.3) node[right] {Research};

      \fill[fill=skyblue1] (0,0) -- (270:1) arc (270:360:1) -- cycle
      node[white,font=\Huge] at (315:0.5) {2};
      \draw (315:1) -- (315:1.3) node[right] {Rapport and trust};

      \fill[fill=scarletred1] (0,0) -- (180:1) arc (180:270:1) -- cycle
      node[white,font=\Huge] at (225:0.5) {3};
      \draw (225:1) -- (225:1.3) node[left] {Exploiting trust};

      \fill[fill=butter2] (0,0) -- (90:1) arc (90:180:1) -- cycle
      node[white,font=\Huge] at (135:0.5) {4};
      \draw (135:1) -- (135:1.3) node[left] {Utilizing information};

    \end{tikzpicture}

    \caption{A tipical social engineering attack cycle.}
  \end{center}
\end{figure}

\subsubsection{Research}

This phase consists of several methodologies to gain an extremely high amount of
information about the target. Different methods might be used, depending on
whether the target is a company or an individual. The ultimate goal of this
phase is to gain as much information as possible about the target to develop a
relationship, rapport and trust in the next phase. Of course, there are
enormous possibilities to get information on companies and individuals, the
most known are described by~\cite{jones2004,mitnick2003,thornburgh2004}, but of
course they are not limited to those.

\begin{description}

\item[Corporate Website] The Website of a company might be the first place to
  start looking out for such information. Lists of employees,
  company directors, or company brochures is often publicly available.
\item[Personal Homepage] If the target is an individual, the personal homepage
  of the target might offer more information than the company website. If the
  target runs a blog, information about the workplace can be accessed
  also quite often. This information can then be used to attack the employer.
\item[Dumpster Diving] This is a technique was and is also used by several
  intelligence agencies~\cite{lively2003}. It includes the search
  for useful information in the rubbish container of companies and individuals. While
  this seems a strange technique, a social engineer still can gain much
  information out of this.
\item[Phishing] Fake e-mails or websites are set up and sent/accessed by
  individuals. They both have the goal to make the target believe that they are
  watching a regular e-mail or website and entering sensitive information.
  Phishing's description can range from a complete social engineering attack to
  a simple information gathering.
\item[Trojan Horses and Other Malware] These tools have the same goal as
  phishing, however they work automatically and often do not require any
  user interaction, except for example installing the software. They can
  be used for other objectives as well, for example a keylogger or a tool
  which logs the website the target is accessing or e-mails.
\item[Newsgroups and Mailing Lists] Several employees or individuals are
  subscribed to public accessible newsgroups and mailing lists. These are often
  technical groups and lists and therefore expose information, like which system
  is in use and how it is configured.
\item[Job Sites] Companies often look to hire new people and many
  times they have to include sensitive information, if they look for
  people who have gained experience on that field. Like above, information about the
  computer systems can also be obtained here.
\end{description}

As social networks are quite new, the existing literature does not count them
yet as a typical research possibility for social engineers. Social networks
however offer a wide range of the above mentioned possibilities. For
example, the corporate website could be replaced if a large amount of the
employees use a specific social network. The personal homepage might be
replaced by a social network, if enough private data about a person is
available on the social network. Social networks also often include newsgroups,
job sites and other information on their platform, making it even easier to
access a large amount of data about a single person.


\subsubsection{Rapport and Trust}

In this second phase, the social engineer makes contact via several ways of
communication. The attacker uses techniques like name-dropping, where he
\textit{drops} several names of verifiable employees or individuals to make the
target believe the attacker is who he pretends to be. Once the attacker has
established himself as authentic, he goes onto the next phase. Jones
shows the following physical settings, in which an attack can happen~\cite{jones2004}:

\begin{description}
\item[Telephone] The most common type of social engineering attacks are those
  made by telephone. Almost every person is vulnerable to this type,
  especially call desks, but also individuals and employees.
\item[Workplace] This is a quite dangerous type, as an attacker has to gain
  physical access to the targets workplace. However, there are many
  possibilities of how a talented attacker can gain legitimate access to a building.
  Once inside, the attacker can look out for passwords, sensitive documents,
  do shoulder-surfing or access computers and network, which are left
  unguarded in the accessed area.
\item[E-mail] The social engineer can compose an original looking document,
  either from the company the target works for, another company or a service,
  the target uses, like a social network.
\item[Chat] Similar as described above, the attacker connects to the victim directly,
  forcing the target to install malicious software or exposing sensitive
  information.
\end{description}

Social networks allow an additional way to create a rapport with the victim,
for example the often integrated chat, forum and messaging functions. The
rapport could be established on a thinner basis, depending on the method used.
However, this could also be a desired effect by the social engineer.


\subsubsection{Exploiting Trust}

Once the connection has been established, the attacker can exploit the trust or
better the relationship of reliance by \textit{\glqq{}weaving a story that plays upon
the emotions of the victim\grqq{}}~\cite{thornburgh2004}. Social engineers
heavily rely on psychology to exploit their trust. Jones lists the
following~\cite{jones2004}:

\begin{description}

\item[Authority] This can be a highly effective psychological weapon, for
  example when impersonating the boss of the company. Orgill et al. have
  proved by means of a real case study that social engineering attacks are even more
  effective when the supervisor supports the attack or an attacker
  impersonates a supervisor~\cite{orgill2004}.
\item[Natural Tendency to be Helpful] Psychology shows that it is a natural
  tendency to be helpful. A social engineer can use this to his advantage.
  Several examples are shown by Mitnick and Simon~\cite{mitnick2003}, for example
  desperate calls in the night to the help desk, or a person trying to get access
  to the building with many boxes in his hand.
\item[Liking and Similarity] A personal connection might be helpful for the
  attacker. Psychology shows that it is natural to like people who are like
  themselves. If information about hobbies or interests are available, a social
  engineer might use that information to establish a relationship based on 
  it. Once this connection has been made the victim may feel more
  responsive to the attacker and therefore might offer more sensitive
  information.
\item[Reciprocation] Social interaction is often based on the fact that if
  someone gives something, one might give him something in return. In social
  engineering, this is called \textit{reverse social engineering}. In this case, a
  social engineer tries to help the victim and in return requests a favour from
  that person.
\item[Low Involvement] If a person has low to no involvement in the company or
  towards a victim, it might have little interest in what the social engineer is
  asking. Commonly, receptionists, the cleaning crew and similar people are
  favourable victims.
\end{description}

These constructs can also be exploited by using the information available on social
networks. For example, the authority might not exist on a social network,
however if the attacker is able to create a large social group around, he surely
has a certain authority he can use against the victim. The gained information
can even help to exploit the trust, for example if an attacker already knows
how helpful the victim is, what the victim likes or dislikes or if he shows low
involvement.

\subsubsection{Utilizing Information}

This phase finally utilizes the gained information to either implement a
technical attack, a non-technical attack or to use the gained information for a
new social engineering attack~\cite{thornburgh2004}. The actual attack and the
threat it exposes is created in this phase. It is interesting that the attack
created here does not limit itself to technical or non-technical attacks, but
can also be used to publish sensitive information or other.

\subsection{Threats and Risks}

Thornburgh~\cite{thornburgh2004} defines a social engineering attack as
successful if the attacker achieves his set objective, even if the information
itself is either not enough to perform penetration or if the information is
considered of not great value by the target. Moreover, every small piece of
information can help the attack to become successful and increases the
possibility of it's success.

Orgill et al.~\cite{orgill2004} suggests that social engineering is an ever-present
threat to the security of computer systems, because it does not attack the
computer itself, but the human being using the computer. It exploits the
natural tendency of humans to trust others, helping them and gain favour.

In an introduction to social engineering, Manske~\cite{manske2000} offers the
following explanation:

\begin{quote}
\textit{Successful social engineering attacks give the attacker the means to bypass
millions of dollars invested in technical and non-technical protection
mechanisms and consulting, completely nullifying security investment,
Firewalls, secure routers, PKI, e-mail\dots and security guards are all down
the drain.}
\end{quote}

The risk of a social engineering attack depends mostly on the value of the
gained information~\cite{thornburgh2004}. If the gained information cannot help
to develop another attack, the impact of this special attack is minimal to
non-existent. On the other side, valuable information can lead to a very
dangerous attack, especially if the attacker remains hidden and the attack has
not been detected throughout his attack. If the attacker is not able to remain
hidden, a system penetration is still possible, however subsequent attacks can
eventually be prevented. Overall, a social engineering attack can be as dangerous
as the information gained. However, non-valuable information can still help the
social engineer to acquire more data.

It was shown that a typical social engineering attack is cyclical and
therefore can be repeated inside a company or by using several individuals. The
target can be penetrated over and over again, until the ultimate goal is
reached. Not to mention the different type of goals that can be reached.
Thornburgh~\cite{thornburgh2004} and Mitnick and Simon~\cite{mitnick2003}
mention that social engineers take care to keep the target for further attacks
and not to \textit{burn} it~\cite{thornburgh2004}. They try to remain anonymous
and do everything to prevent their detection.

Amongst several others, Mitnick and Simon~\cite{mitnick2003}, Winkler and
Dealy~\cite{winkler1995} and Orgill et al.~\cite{orgill2004} show how easy it
is to get access to a company. They use several methods, like telephone,
surveys and other methods. All were extremely successful and while not being
expensive, all of the case studies managed to get sensitive information in a
very short amount of time, despite strong security measures. Additionally all
of these did not use very company specific data and the case studies can easily
be extrapolated to other attacks at other companies.

\subsection{Countermeasures}

Winkler and Dealy~\cite{winkler1995} prove that many of the weaknesses
exploited by the attackers were common in most companies. Although they mention
that even the best technical mechanisms had not prevented the attacks. The
attacker exploited poor security awareness and even if the attackers had not
been able to get passwords, they still would have been able to acquire
sensitive personal and company information.

Moreover, security operators have to consider the non-technical side of
computer security and not assume that for example cryptography is enough to
prevent such attacks. In addition, computer professionals believe all too often that
computer security fundamentals are known to everyone. Lively~\cite{lively2003} hits
the point with following: \textit{\glqq{}Users cannot defend against what
they do not know\grqq{}}. Most of the related work see user awareness and user
rewards as an essential part against social engineering, however they agree that
there is no way to protect themselves against social engineering attacks.

Mitnick and Simon~\cite{mitnick2003} give some indication of how a social
engineering attack can be recognized. However, all of these points still can
happen in a totally ordinary phone call or dialogue. Even so, they give a good
base to recognize attacks.

\begin{itemize}
  \item Refusal to give a callback number
  \item Out of ordinary request
  \item Claim of authority
  \item Stresses urgency
  \enlargethispage{\baselineskip}
  \item Threatens negative consequences
  \item Shows discomfort when questioned
  \item Name dropping
\end{itemize}

Winkler and Dealy~\cite{winkler1995} try to create a more concrete policy with
the following points:

\begin{description}
  \item[Do not Rely upon Common Internal Identifiers] Some companies use basic
  authentication methods to authenticate as real employees. Unfortunately the
  employee number or similar identifiers are commonly used by companies.
  Therefore it is quite easy to get a list or just some single identifiers and
  use them for authentication. The authors suggest having a different
  identifier for computer related activities and not to rely upon such simple
  identifiers.
  \item[Implement a Call Back Procedure] Often, social engineering attacks can
  be prevented if employees would have verified the caller identity by calling
  them back at their proper number, as listed in the company directory. This of
  course also includes e-mail or instant messaging. Although it creates an
  inconvenience, but when compared to the potential losses, this is of course
  justified, as the team shows. Caller ID services or reverse DNS services can
  be helpful too.
  \item[Implement a Security Awareness Program] An enormous amount of money is
  often spend by companies on hard- and software security devices. Still, security
  awareness programs are barely implemented. As stated before, computer
  professionals cannot assume that security practises, even the simplest, are
  known to every user. The paper also mentions that a good security awareness
  program can be implemented for minimal cost and can save a lot of money
  compared to the losses.
  \item[Identify Direct Computer Support Analysts] In a company, every employee
  should be familiar with a single well known computer analyst. The analyst on
  the other side should not have more than 60 users. He ought to be the only
  person who can contact and should be contacted by the users. The users
  shall also contact their computer analyst immediately if another person
  from the computer support contacts them.
  \item[Create a Security Alert System] During the attacks the authors 
  recognized that there was no way to warn other employees if a single
  employee of the company would have detected them. So, the attack can
  continue until the attacker's goal is reached, even if they were detected by some
  employees. They also mention that a detection would only have improved the
  attack, as the attackers would have learned what works and what does
  not.
  \item[Test Security Policies] Many people already test their physical
  and electronical security devices, however, the human vulnerabilities is often not
  dealt with. To enable the above mentioned policies, they have to be tested
  by social engineering attacks for their effectiveness. And still, those
  policies cannot assure 100\% protection against social engineering attacks.
\end{description}

\section{Social Networks}

Social networks gained millions of members on their platforms in the last years.
They are widely used in both private and business networking. Social networks
allow individuals to present their own identity and to share and keep
relationships with other members of those services.

As these platforms became so popular in the last years and store an enormous
amount of data about each user, the question must be raised what the
relationship between the classic social engineering and the new technology of
social networks. It also raises the question about new threats and attacks
which are possible by either just using social networks as an attack platform
or by using social networks as an information pool. However, both together
seem to be possible too.

Many studies, as for example~\cite{fraunhofer2008,gross2005} already show that privacy
and information revelation in social networks is a massive problem. The following
sections want to take a deeper approach and characterize what actual security
issues social networks disclose. These attacks can be characterized as new
threats, which did not exist before. 

The related work can be divided in two parts: \textit{active} attacks and
\textit{passive} attacks, whereas \textit{active} attacks contain attacks
which exploit services and users of social networks, like messaging services,
birthday invitations and other. \textit{Passive} attacks mean that the attack
is driven passively by harvesting data and eavesdropping, commonly without having
access to the network. It is important to note that the active attacks are
rather real attacks, whilst the passive attacks are more of an information
gathering, on which an attack might follow.

\subsection{Active Attacks}
\label{subsection:active_attacks}

Jagatic et al.~\cite{jagatic2007} ran a phishing attack against a number of
students of the Indiana University, aged between 18 and 24 years. They
harvested freely available information on social networks and built a dataset.
They then spoofed an e-mail message between two friends on a social network
using the first person as a sender and the second as the recipient. The message
contained a phishing site clearly marked as a phishing URL, which then asked
the recipient to enter his university credentials. By means of a control group,
they then sent the same message with an unknown sender. The result showed that
about 72\% of all students did enter their university credentials on the
insecure site, while only 16\% of the control group did the same. The authors
mentioned that the result was much higher than anticipated, as there were many
ways to detect the phishing attack, as for example the non-university URL of
the site, a bogus authentication message, the \textit{WHOIS} entry and others.
However many students did change the password afterwards and installed anti
virus software, as they assumed malware on their computers. This is of course
useless, as the attackers have already obtained access.

A more concrete attack was carried out by Brown et al.~\cite{brown2008}. The
researchers did get sample data out of a popular group on the \textit{Facebook}
social network. They then analyzed the publicly available attributes of each
group member, trying to determine how they could be attacked. They described
three types of attacks, which they actually carried out on \textit{Facebook},
but are transferable to other social networks as well.

\begin{description}

\item[Relationship-based Attacks] Attacks in this category do only use the
relationship information of several users for the attack. No attributes other
than the relationship are used. The attack described by Jagatic et
al.~\cite{jagatic2007} falls into this category. As an additional example
attack, they spoofed a message notification that looked like an official one
that was sent by an actual friend of the recipient and contained a phishing
URL.

\item[Unshared-attribute Attacks]
These are attacks which use attributes of only one user together
with the relationship information to carry out the attack. The authors created
a sample attack, in which they either sent a birthday greeting card or a
birthday invitation on the actual birthday date of a user to him or to his
friends, similarly containing a phishing URL.

\item[Shared-attribute Attacks]
As mentioned previously, these attacks use the relationship information, together with
attributes, which are shared by both users. As a sample attack they created
a message again, containing a link to a photo site where photos of a common
event are located. Again, the URL in the message pointed to a phishing site.
\end{description}

The results of the attacks mentioned above are listed in table
\ref{tab:context-aware-spam}, containing the three types of attacks based on
profile openness. Furthermore, they estimated that about 85\% of the users
can be accurately targeted by such attacks and even users with closed
profiles or strict privacy settings are almost equally vulnerable.

\begin{table*}[ht]
  \centering
  \ra{1.3}
  \begin{tabular}{lccc}
    \toprule
    & Open profiles & Close profiles & All\\
    \midrule
    Relationship-only attacks & 85\% & 84\% & 85\% \\
    Birthday greeting         & 74\% &  0\% & 50\% \\
    Birthday invitation       & 84\% & 84\% & 84\% \\
    \bottomrule
  \end{tabular}
  \caption{Results of the case study of Brown et al.~\cite{brown2008}}
  \label{tab:context-aware-spam}
\end{table*}

\subsection[Passive Attacks]{Passive Attacks (based on Gross et al.~\cite{gross2005})}

Information revelation is one of the most important entities in passive attacks,
as they provide a good information source for further attacks. Gross et al.
analyzed \textit{Facebook} and showed potential attacks by just using the
information presented on profile pages~\cite{gross2005}. This paper is one
of the first that describes passive attacks in social networks.

The study shows that \textit{Facebook}, and probably other social networks
too, provide an enormous amount of information about their users. For example,
90.8\% of the profiles of the dataset they acquired contained a picture, 87.8\%
revealed their birth date, 50.8\% their current residence. Furthermore, most of
the users convey information like their dating preferences, current relationship
status and the name of their partner, political views, interests, jobs and many
more. They show that just by viewing profiles, one is able to connect first
and last name, their residence and birth date.

Regarding the validity of the data, they found that 89\% of all names may
likely be real names, whilst 8\% did use a fake name. The remaining 3\% just
disclosed their first name. Even if they are not forced to provide their full
names, the vast majority does so. Looking at other data, around 98.5\% disclosed
their quite identifiable birth date, 61\% did provide a thorough identifiable
picture of themselves, 19\% were semi-identifiable and 8\% did provide a group
image of altogether 90.8\% users who provided a picture.

Furthermore, just a small number actually did change the default privacy
preferences and therefore their fully identifiable information is available to
every user registered at \textit{Facebook}.

The possibilities of attacks based on the data mentioned above are endless,
however the authors outlined a few. Basically, they all state that a large
amount of data about a single person is available and accessible without
authorization by that person. Possible attacks contain stalking,
cyber-stalking, demographics re-identification, face re-identification, theft
of social security numbers and identity theft.

The authors claim that there is a large amount of people who are willing to
provide large amounts of personal information, while the user's notion relating
to privacy risks is often unconcerned. Furthermore, they show that they expose
themselves to various risks and make it extremely easy to create records about
them. In addition, those risks are not unique to \textit{Facebook}.
