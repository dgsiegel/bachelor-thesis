\chapter{Evaluation}
\label{chap:evaluation}

This chapter will try to evaluate the new threats disclosed by the prototype.
First, some sample attacks are driven and then analyzed. In the end, this
chapter will try to extrapolate the possibility of further social engineering
attacks.

\section{Demonstration of Attacks Using the Prototype}

This section wants to demonstrate how the prototype, which was
developed in the previous chapter, can assist a social engineering attack.
Therefore, the attacks previous described in chapter \ref{chap:attacks} are
repeated together with the aid of the prototype.

\section{Phishing Mail}

The attack itself was already described in section \ref{sec:phishing_mail}. It
will now be driven against a sample profile on \Twitter{} using the prototype.
The attack is visually represented in figure \ref{fig:evaluation_phishing}.
Following data is needed:

\begin{itemize}
  \item Real name of the victim
  \item E-mail address of the victim
  \item The Knowledge, that the victim is a customer of an electronic
  payment service, like \textit{PayPal}, \textit{Amazon} or \textit{eBay}
\end{itemize}

The attacker now simply starts off with a simple keyword search, as he is not
interested in a specific person (at least for now). Therefore he runs:

\lstset{language=bash}
\begin{lstlisting}
$ python prototype.py -k "paypal"
\end{lstlisting}

This opens a new website on \url{http://search.twitter.com}, which shows the
attacker people, who quoted the word \textit{paypal} in one of their messages.
The attacker now picks a person with the username \textit{petersample}, who writes

\begin{quote}
did really some transactions over paypal this week, works really great!
\end{quote}

As the attacker now has an individual he can attack, he is interested in more
information. Therefore he runs:

\lstset{language=bash}
\begin{lstlisting}
$ python prototype.py -u "petersample"
\end{lstlisting}

This now produces a fact sheet about Peter Sample. As the attacker already
knows, that the victim is using \textit{PayPal}, he just needs the real name
and the e-mail address. The output of the prototype is displayed in figure
\ref{fig:prototype_user}. The attacker now knows the real name. The e-mail
address was also found by the prototype and is displayed in figure
\ref{fig:prototype_mail}. Since every data needed for the phishing mail is
acquired, the attacker can begin to send his e-mails out. As described,
it is also possible to include Peter Sample's friends, if they are using
\textit{PayPal} too. This is simply done by redoing the attack on Peter
Sample's friends, which the prototype also outputs.

\begin{figure}[htb]
  \begin{center}
    \includegraphics[width=\textwidth]{prototype_user}
    \caption{Prototype output: general information about a user.}
    \label{fig:prototype_user}
  \end{center}
\end{figure}

\begin{figure}[htb]
  \begin{center}
    \includegraphics[width=\textwidth]{prototype_mail}
    \caption{Prototype output: e-mail address was found.}
    \label{fig:prototype_mail}
  \end{center}
\end{figure}

\begin{figure}[ht]
  \begin{center}
    \includegraphics[width=0.75\textwidth]{evaluation_phishing.1}
    \label{fig:evaluation_phishing}
    \caption{Activity diagram of the phishing attack.}
  \end{center}
\end{figure}

\section{Insider Attack}

In the attack described in section \ref{sec:insider_attack}, the attacker
starts again with a keyword search, as he wants to find an employee of Sample
Company Inc. Quite quickly he finds \textit{petersample}, who seems to work for
one particular company. He launches an information retrieval about that user
and gets an output like in figure \ref{fig:prototype_user2}.

\begin{figure}[htb]
  \begin{center}
    \includegraphics[width=\textwidth]{prototype_user2}
    \caption{Prototype output: general information about an employee of Sample
    Company Inc.}
    \label{fig:prototype_user2}
  \end{center}
\end{figure}

He now knows the real name, the company and department Peter Sample works for,
the workgroup inside the company, and has an approximate idea where the victim
could live. A text search about phone specific terms, produces the output
shown in figure \ref{fig:prototype_phone}. The employee number is
trickier to get, as one must rely that the employee posts this number in a
message. In this scenario, this is not the case and the social engineer has to
call Peter Sample at his phone number.

\begin{figure}[htb]
  \begin{center}
    \includegraphics[width=\textwidth]{prototype_phone}
    \caption{Prototype output: text search about phone specific terms.}
    \label{fig:prototype_phone}
  \end{center}
\end{figure}


\begin{figure}[htb]
  \begin{center}
    \includegraphics[width=\textwidth]{prototype_residence}
    \caption{Prototype output: the residence is revealed.}
    \label{fig:prototype_residence}
  \end{center}
\end{figure}

The prototype also shows the times Peter Sample mostly writes his updates, as
shown in figure \ref{fig:prototype_times}, and therefore is most probably
available, and not having holidays for example. This was also done in the
original scenario, however this time much less information is needed, as most
data can already be retrieved from the social network. Next, he needs the residence
of Peter Sample, which the prototype also supports and is displayed in figure
\ref{fig:prototype_residence}.

\begin{figure}[htb]
  \begin{center}
    \includegraphics[width=0.49\textwidth]{prototype_times_month} 
    \includegraphics[width=0.49\textwidth]{prototype_times_day} 
    \includegraphics[width=0.49\textwidth]{prototype_times_hour}
    \caption{Prototype output: the average times a user writes his updates: months,
    week, day average.}
    \label{fig:prototype_times}
  \end{center}
\end{figure}

The use of security devices and server names is either determined by the phone
call mentioned above or by a text search. A text search works quite good, as
many people write some message about private details things, which are not
working at the moment, like a broken connection to their development server or
that they dislike special security devices. This data can be harvested like the
phone number before.

The company structure, like the managers name and colleagues of the employee
can be determined by analyzing the hidden friends network and the replies the
employee made. By further analyzing the friend's networks, replies and applying
the same data retrieval as above, a quite detailed company and employee
structure can be determined.

As in the original scenario, the attacker now just has to wait for a snow storm
in the area of the residence of the employee, which is easy to check, as he
has the residence address.

\section{The Bank Heist}

The scenario described in section \ref{sec:bank_heist} is quite different from
the others, as it requires data, which normally is not or at least should not
be published. While the first part of the scenario, like names, phone numbers
and departments are possible to extract either by using the prototype or by
directly exercising social engineering attacks, the daily wire transfer code is not
published. While it still might be possible for a social engineer to get the
code, it is not possible to do that over a social network like \Twitter{}, as
this sensitive data is not published. Though, the prototype can still save a
lot of effort, for example by determining the structure of the company or
analyzing company employees.


\section{General Feasibility and Extrapolation of Attacks}

Attacks and information retrieval are possible with the prototype. That was
shown in the above scenarios. One big barrier though remains private and
sensitive data, at least if a user of a social network identifies specific data
as sensitive. However many users do not, and post e-mail addresses, residence
addresses, phone numbers, passwords and further more.

Harvesting this data automatically makes the life of a social engineer
definitely easier, as he just has to execute the attack and is no more
constrained to do an information research beforehand. When it comes to
sensitive data, a prototype like the one introduced, can still help to gain
data, like company information or user specific information.

With it's plugin architecture, the prototype is easily extendible and more data
and analysis can simply be done. For this prototype the \Twitter{} social
network was chosen, however other famous social networks are possible too.
Putting together more social networks and possibly search engines too, one can
get an even more complete picture of a victim.

Most of the effort put into the prototype was to do text mining. Though, there is
also a lot of optimization possible regarding text analysis. However, the
prototype was able to gain much more information, than a simple or even a more
complex text search engine could do, for example a web search engine. The prototype
offers a wide range of analysis possibilities, all suited for special social
engineering attacks. This was shown in the previous sample attacks.

The prototype did get all data legally and can operate almost anonymously, as
he won't be tracked, with the small amount of data it fetches. Though it has
a big effect and can lead to very dangerous attacks, which can be carried out
relatively easy.

\section{New Threats}

Critical attacks based on social engineering techniques are possible, that was
demonstrated in this work. The question one should pose here is, what new
threats this work did discover. As already stated before, the threat is not
generated from the possibility to gather information, it is brought forth by
social engineering attacks, which are based on such information. The work
showed, that it is possible to gather large amount of sensitive data of several
users fast and almost anonymously. This means, that the research phase of a
social engineering attack can be almost replaced and done automatically,
depending on the attack scenario. By fetching the data automatically and
without knowledge of the victim, the attacker can prepare the attack while
remaining hidden and not exposing himself to the danger of being discovered.

Taking this approach one step further, automatic harvesting and attacks could
be a future scenario. An attacker could define some social engineering attacks,
like the phishing mail attack, which was shown before, and let them run over a
social network. As there is no way to stop such an attack, this would have an
enormous impact. Brown et al.~\cite{brown2008} for example were able to create
a context-aware phishing mail attack inside a social network, which was
extremely successful. This scenario however is not just restricted to phishing
mail attacks, but is open to all social engineering attacks, where the attacker
no longer chooses a suitable attack for a victim, but the victim chooses the
attack by himself, by just exposing sensitive data.
