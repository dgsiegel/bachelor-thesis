\chapter{Selection of Social Engineering Attacks}
\label{chap:attacks}

The previous chapter showed already sample social engineering attacks, the
threats it exposes and their countermeasures. The work wants to assign now selected
social engineering attacks to social networks and find out what new threats
they disclose. Furthermore, the attacks should constitute a part of the demand
of the prototype, which will be developed.

All of the following attacks are described by Mitnick and Simon
\cite{mitnick2003} and have happened in the past.

\section[Phishing Mail]{Phishing Mail (\cite[pp. 97-100]{mitnick2003})}
\label{sec:phishing_mail}

The first attack is an attack against a private person, with no affiliation at
a company. A phishing mail is used to trick the victim into revealing his
username and password of an online payment service. A definition of the term
\textit{phishing} is as follows~\cite{jagatic2007}:

\begin{quote}
\textit{Phishing is a form of deception in which an attacker attempts to
fraudulently acquire sensitive information from a victim by impersonating a
trustworthy entity. Phishing attacks typically employ generic
\glqq{}lures\grqq{}.}
\end{quote}

The enormous threat that phishing attacks can present was already shown in
section \ref{subsection:active_attacks}. The following scenario was therefore
chosen because of its relevancy and the fact that just individuals are being
attacked.

The victim, a retired insurance salesman named Edgar, received an e-mail from
\textit{PayPal}\footnote{\url{http://www.paypal.com}}, which is a company that
offers a fast and secure way to make online payments. Edgar did use
\textit{PayPal} often, as he was a collector of antique glass jars. Therefore
he used the service several times a week. The e-mail he received during the
holiday season 2001 was looking officially from \textit{PayPal}, offering him a
requital for updating his \textit{PayPal} account. The message looked like the
following~\cite[p. 97]{mitnick2003}:

\begin{quote}
Season's Greetings Valued PayPal Customer;

As the New Year approaches and as we all get ready to move a year ahead,
PayPal would like to give you a \$5 credit to your account!

All you have to do to claim your \$5 gift from us is update your information on
our secure Pay Pal site by January 1st, 2002. A year brings a lot of changes,
by updating your information with us you will allow for us to continue
providing you and our valued customer service with excellent service and in the
meantime, keep our records straight! To update your information now and to
receive \$5 in your PayPal account instantly, click this link:

\url{http://www.paypal-secure.com/cgi-bin}

Thank you for using PayPal.com and helping us grow to be the largest of our
kind!

Sincerely wishing you a very "Merry Christmas and Happy New Year,"

PayPal Team

\end{quote}

One might notice several signs in the message, that can lead to the belief,
that something is wrong with the e-mail. For example, the semicolon after the
greeting line, the blemished text about 
\glqq{}our valued customer service with excellent service and in the\grqq{} and
most noticeable the URL, which does not lead to \url{http://www.paypal.com} but
to a different domain. Edgar clicked on the link, entered the information
requested, which was name, address, phone number and credit card number. Then,
he was waiting for the \$5 gift to show up, but what showed up was a list of
charges he never purchased on his credit card bill.

The victim is not a single one, there are many scams which look like the above.
The attack was also prepared properly, as they knew that Edgar was a PayPal
customer. If he would not have been, the attack certainly wouldn't have worked.

To make the attack even more realistic and effective, the message could have also
been sent to the victim's friends and colleagues, if they are also a member of
the same electronic payment service.

\begin{table*}[ht]
  \centering
  \ra{1.3}
  \begin{tabular}{p{0.8\textwidth}c}
    \toprule
    Information & Required\\
    \midrule
    Real name & \checkmark\\
    E-mail address & \checkmark\\
    Knowledge of account of an electronic payment service & \checkmark\\
    E-mail addresses of friends and colleagues & \\
    Knowledge, which friend or colleague has an account on an electronic
    payment service & \\
    \bottomrule
  \end{tabular}
  \caption{Overview of the required data of the phishing attack.}
\end{table*}

\section[Insider Attack]{Insider Attack (\cite[pp. 83-89]{mitnick2003})}
\label{sec:insider_attack}

Another real case scenario was an attack, which involved stealing source code
from a company containing the encryption algorithms and firmware used in the
company radio products. Its relevance is the fact that this was an attack
against a company, involving information gathering about the company structure,
security devices used inside the company and having several employees as
victims. Due to its manifold character, this attack could be extended to
similar attacks which even could require less information.

The attacker, called \textit{Danny} by Mitnick, began his information retrieval
on the Internet. He luckily found a years-old message written by an employee of
the affected company and posted to a public readable newsgroup. This message
contained a signature including the employee's name, his phone number and
workgroup. He now had to check if that person still works for that company. He
called the employee, who indeed was still working for the same company, and
manipulated him to reveal the names of the servers the employees used for
development work.

In addition, every employee in the company had a small electronic device called
\textit{Secure ID} in addition to their username and password. The attacker
managed to get enough pieces of information about the company together to
masquerade as a real employee. He now had an employee's name, residence
address, phone number, department and employee number, the manager's name and
phone number. Furthermore he knew the servers he needed access for.

Danny now waited for a snow storm in the location of the employee's residence.
As it was winter, he did not have to wait very long and he launched the attack.
He telephoned the IT department and talked to a computer operator named Roger
Kowalski.

The attack went as follows~\cite[p. 86]{mitnick2003}:

\begin{quote}
\textit{Danny:} \glqq{}This is Bob Billings. I work in the Secure Communications Group. I'm at
home right now and I can't drive in because of the storm. And the problem is
that I need to access my workstation and the server from home, and I left my
Secure ID in my desk. Can you go fetch it for me? Or can somebody? And then
read off my code when I need to get in? Because my team has a critical deadline
and there's no way I can get my work done. And there's no way I can get to the
office--the roads are much too dangerous up my way.\grqq{}\\
\textit{Roger Kowalski:} \glqq{}I can't leave the Computer Center.\grqq{}\\
\textit{Danny:} \glqq{}Do you have a Secure ID yourself?.\grqq{}\\
\textit{Roger Kowalski:} \glqq{}There's one here in the Computer Center, we keep one
for the operators in case of an emergency.\grqq{}\\
\textit{Danny:} \glqq{}Listen, can you do me a big favor? When I need to dial into
the network, can you let me borrow your Secure ID? Just until it's safe to
drive in.\grqq{}\\
\textit{Roger Kowalski:} \glqq{}Who are you again? Who do you work for.\grqq{}\\
\textit{Danny:} \glqq{}For Ed Trenton.\grqq{}\\
\textit{Roger Kowalski:} \glqq{}Oh, yeah, I know him.\grqq{}\\
\textit{Danny:} \glqq{}I'm on the second floor, next to Roy Tucker.\grqq{}\\

\end{quote}

The operator of course was uncomfortable walking into the office desk and
looking through foreign property. But he was uncomfortable not helping either,
so he asked his manager and actually vouched for the attacker. The manager
wanted to speak to the attacker personally and the operator gave him the name
and phone number.

\begin{table*}[h]
  \centering
  \ra{1.3}
  \begin{tabular}{p{0.8\textwidth}c}
    \toprule
    Information & Required\\
    \midrule
    \multicolumn{2}{l}{Colleagues}\\
    \hspace{0.5cm} Name & \checkmark\\
    \hspace{0.5cm} Phone number & \checkmark\\
    \hspace{0.5cm} Department & \checkmark\\
    \hspace{0.5cm} Employee number & \\
    \multicolumn{2}{l}{Manager}\\
    \hspace{0.5cm} Name & \checkmark\\
    \hspace{0.5cm} Phone number & \\
    Residence of an employee & \checkmark\\
    Weather information & \checkmark\\
    Company and employee structure & \checkmark\\
    The knowledge, what security devices were used in the company & \checkmark\\
    The servers names and location & \checkmark\\
    \bottomrule
  \end{tabular}
  \caption{Overview of the required data of the insider attack.}
\end{table*}

The attacker then called the manager and explained the same story again,
mentioning that his team has a critical deadline. The manager then allowed him
to use the \textit{Secure ID} device in the IT department just for the weekend.
From now on, Danny just needed to call in and ask the operators to pass him the
\textit{Secure ID} token. Furthermore he was able to get a temporary account to
bypass the firewall restrictions, directly from the operator. Now, he had the
whole weekend to find a security hole which he found.

\section{The Bank Heist}
\label{sec:bank_heist}

The last attack was chosen due to the fact that it uses information, which is
not available on any social networks or the Internet. Again, this is an attack
against a company, including information about several employees, the company
structure and the already mentioned secret information.

\begin{table*}[ht]
  \centering
  \ra{1.3}
  \begin{tabular}{p{0.8\textwidth}c}
    \toprule
    Information & Required\\
    \midrule
    \multicolumn{2}{l}{Employee of the wire transfer room}\\
    \hspace{0.5cm} Name & \checkmark\\
    \hspace{0.5cm} Employee number & \checkmark\\
    \multicolumn{2}{l}{Employee outside the wire transfer room}\\
    \hspace{0.5cm} Name & \checkmark\\
    \hspace{0.5cm} Department & \checkmark\\
    \hspace{0.5cm} Employee number & \checkmark\\
    Phone number of wire transfer room & \checkmark\\
    Current wire transfer code & \checkmark\\
    \bottomrule
  \end{tabular}
  \caption{Overview of the required data of the phishing attack.}
\end{table*}

The bank heist was executed in 1978 by Stanley Mark Rifkin. It was the largest bank
robbery in U.S. history by that time. He stole 10.2 million U.S. dollars
through wire transfer just by using a telephone and social engineering
techniques.

Rifkin was working for a company under contract developing a backup system for
the Security Pacific National Bank and more specific for the wire transfer
system. As he was working inside the wire transfer room, he had the possibility
to learn the process of bank wire transfer. As the bank changed the
required transfer code daily, most employees did write the code down in order
to not have to memorize it every day. He then tried to adept the names, room
numbers and employee numbers. One day he additionally memorized the daily code and
executed a wire transfer over a callbox outside of the bank. Rifkin transfered
over 10 million U.S. dollars to a bank in Switzerland by just using the
information he already had gained and social engineering techniques.

He then picked up the money in Switzerland, changed it into diamonds and flew
back to the United States. As he tried to sell them there, the FBI quickly
captured him.
