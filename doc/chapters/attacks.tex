\chapter{Concrete Social Engineering Attacks}
\label{chapter:attacks}
\comment{add a better chapter title}

The previous chapter showed already sample social engineering attacks, the
threats it exposes and countermeasures. Now, a few social engineering attacks
are going to be introduced. All of those are described by \cite{mitnick2003}
and are happened in the past.

\section{Phishing mail}

The victim, a retired insurance salesman named Edgar, received an email from
\textit{PayPal}\footnote{\url{http://www.paypal.com}}, which is a company that
offers a fast and secure way to make online payments. Edgar did use
\textit{PayPal} often, as he was a collection of antique glass jars. Therefore
he used the service several times a week. The email he received during the
holiday season 2001 was looking officially
from \textit{PayPal}, offering him a requital for updating his
\textit{PayPal} account. The message looked like the following:

\begin{quote}
Season's Greetings Valued PayPal Customer;

As the New Year approaches and as we all get ready to move a year ahead,
PayPal would like to give you a \$5 credit to your account!

All you have to do to claim your \$5 gift from us is update your information on
our secure Pay Pal site by January 1st, 2002. A year brings a lot of changes,
by updating your information with us you will allow for us to continue
providing you and our valued customer service with excellent service and in the
meantime, keep our records straight! To update your information now and to
receive \$5 in your PayPal account instantly, click this link:

\url{http://www.paypal-secure.com/cgi-bin}

Thank you for using PayPal.com and helping us grow to be the largest of our
kind!

Sincerely wishing you a very "Merry Christmas and Happy New Year,"

PayPal Team

\end{quote}

One might notice several signs in the message, that can lead to the belief,
that something is wrong with the email. For example, the semicolon after the
greeting line, the blemished text about 
\glqq{}our valued customer service with excellent service and in the\grqq{} and
most noticeable the URL, which does not lead to \url{http://www.paypal.com} but
to a different domain. Edgar clicked on the link, entered the information
requested, which was name, address, phone number and credit card number. He
then was waiting for the 5\$ gift to show up, but what showed up was a list of
charges he never purchased on his credit card bill.

The victim is not a single one, there are many scams which look like the above.
The attack was also prepared properly, as they knew that Edgar was a PayPal
customer. If he would not have been, the attack certainly wouldn't have worked.

To use this attack, the attacker has to gain access to the following data: 

\begin{itemize}
  \item Real name of the victim
  \item Email address of the victim
  \item The Knowledge, that the victim is a customer of an electronic payment
  service, like \textit{PayPal}, \textit{Amazon} or \textit{eBay}
\end{itemize}

To make the attack even more realistic and effective, the message could also
been sent to the victims friends and colleagues, if they are also a member of
the same electronic payment service.


\section{Insider attack}

This attack was a real case scenario, involving stealing source code from a
company containing the encryption algorithms and firmware used in the company
radio products.

The attacker, called \textit{Danny} by Mitnick, began his information retrieval
on the internet. He luckily found a years-old message written by an employee of
the affected company and posted to a public readable newsgroup. This message
contained a signature containing the employee's name, his phone number and
workgroup. He now had to check if that person still works for that company. He
called the employee, who was still working for the same company, and
manipulated him to reveal the names of the servers, the employees used for
development work.

In addition, every employee in the company had a small electronic device called
Secure ID in addition to their username and password. The attacker managed to
get enough pieces of information about the company together to masquerade as a
real employee. He now had an employee's name, residence address, phone number,
department and employee number, the manager's name and phone number.
Furthermore he knew the servers he needed access for.

Danny now waited for a snow storm in the location of the employee's residence.
As it was winter, he did not have to wait very long and he launched the attack.
He telephoned the IT department and talked to a computer operator named Roger
Kowalski.

The attack went something like this:

\begin{quote}
\textit{Danny:} \glqq{}This is Bob Billings. I work in the Secure Communications Group. I'm at
home right now and I can't drive in because of the storm. And the problem is
that I need to access my workstation and the server from home, and I left my
Secure ID in my desk. Can you go fetch it for me? Or can somebody? And then
read off my code when I need to get in? Because my team has a critical deadline
and there's no way I can get my work done. And there's no way I can get to the
office--the roads are much too dangerous up my way.\grqq{}\\
\textit{Roger Kowalski:} \glqq{}I can't leave the Computer Center.\grqq{}\\
\textit{Danny:} \glqq{}Do you have a Secure ID yourself?.\grqq{}\\
\textit{Roger Kowalski:} \glqq{}There's one here in the Computer Center, we keep one
for the operators in case of an emergency.\grqq{}\\
\textit{Danny:} \glqq{}Listen, can you do me a big favor? When I need to dial into
the network, can you let me borrow your Secure ID? Just until it's safe to
drive in.\grqq{}\\
\textit{Roger Kowalski:} \glqq{}Who are you again? Who do you work for.\grqq{}\\
\textit{Danny:} \glqq{}For Ed Trenton.\grqq{}\\
\textit{Roger Kowalski:} \glqq{}Oh, yeah, I know him.\grqq{}\\
\textit{Danny:} \glqq{}I'm on the second floor, next to Roy Tucker.\grqq{}\\

\end{quote}

The operator of course was uncomfortable walking into the office desk and
looking through foreign property. But he was uncomfortable not helping either,
so he asked his manager and actually vouched for the attacker. The manager
wanted to speak to the attacker personally and the operator gave him the name
and phone number.

The attacker then called the manager and explained the same story again,
mentioning that his team has a critical deadline. The manager then allowed him
to use the Secure ID device in the IT department just for the weekend. From now
on, Danny just needed to call in and ask the operators to pass him the Secure
ID token. Furthermore he was able to get a temporary account to bypass the
firewall restrictions, directly from the operator. Now, he had the whole
weekend to find a security hole which he found.

For this attack, the following data was needed

\begin{itemize}
  \item Name, phone number, department, employee number of several employee's
  \item Name and phone number of the manager
  \item Residence of a employee
  \item Weather information
  \item Company and employee structure
  \item The knowledge, what security devices were used in the company
  \item The servers names and location
\end{itemize}


\section{The bank heist}

not sure if I want this example, however here goes an example which is not
doable with the prototype
