\chapter{Introduction}
\label{chapter:introduction}

\section{Problem and Motivation}

Social Networks, such as Facebook\footnote{\url{http://www.facebook.com}},
MySpace\footnote{\url{http://www.myspace.com}} or
Twitter\footnote{\url{http://www.twitter.com}} gained millions of members on
their platforms in the last years. They are widely used in both private and
business networking. Also, they allow individuals to present their own identity
and to share and keep relationships to other members of those services. Often,
they are also presenting their information to an unknown number of strangers.

This does not only raise the question how useful a social network is for a
single user, but also if there are any drawbacks or risks. Users often put
much and frequently sensitive data into their profiles, as for example
\cite{brown2008} shows. This data however is stored centrally at the company
which provides the network and so the user looses the control over his
data~\cite{fraunhofer2008}.

There is of course a relation between the data the user entered and the user
himself. Following that thought, another person is not only able to extract the
data but also to obtain new information or rather interpretations, which were
not entered by the user.

Many studies, like \cite{fraunhofer2008,gross2005} show already, that this can
be a massive intervention into one's privacy. This work however wants to examine
whether such information are a security risk and can be used for attacks
against a company or an individual.

Of course, a distinction between employees of a certain company and private
users has to be made. For example, certain information can be harmless to an
individual, but very well a danger to a company \cite{mitnick2003}. In both
cases, however, a social engineer saves an additional step, for example a
telephone call or similar, to get that specific information. That step does not
only require extra work, but also the danger of being exposed. Without his
anonymity, a social engineer could not carry out an attack \cite{mitnick2003}.

Especially in big corporations spend hundreds of thousands or dollars for the
security of their IT-infrastructure. An attack at the IT-layer would therefore
often be very laborious. Therefore it is much simpler bypass the security
mechanisms through social engineering, as it very cheap and does not require
any superior technology \cite{winkler1995}. Commonly, a skilled social
engineering does not require anything more than a telephone for such attacks
\cite{mitnick2003}.

Concretely, this study wants to answer the following questioning:
How can data of individuals or companies automatically be extracted
from social networks and presented in a way, that it can be used for a social
engineering attack. In addition, countermeasures against automatic extraction
and social engineering attacks are going to be developed.

\newpage
\section{Outline of the thesis}

\noindent {\textbf{Chapter \ref*{chapter:introduction}}: Introduction}
\vspace{0.5em}\\
\noindent This chapter presents an overview of the thesis and introduces the
reader to the topic of new threats of social networks by exploiting social
networks.

\noindent {\textbf{Chapter \ref*{chapter:relatedwork}}: Related Work}
\vspace{0.5em}\\
\noindent The section will outline related works in the field of social
engineering, the attacks and countermeasures. It will present common attacks
and how companies and individuals can protect themselves against such attacks.

\noindent {\textbf{Chapter \ref*{chapter:attacks}}: Selection of Social Engineering Attacks of the social network \Twitter}
\vspace{0.5em}\\
\noindent Three social engineering attacks, which happened for real, are chosen
and being presented to the reader. They are analyzed in order to let the
prototype repeat the attacks.

\noindent {\textbf{Chapter \ref*{chapter:analysis}}: Analysis of the social network \Twitter}
\vspace{0.5em}\\
\noindent The study takes a deeper look at the social network \Twitter. It will
show the drawbacks and risks of such social networks, the data, which can be
extracted automatically, how sample attacks could look like and finally the
countermeasures against such attacks.

\noindent {\textbf{Chapter \ref*{chapter:prototype}}: Design, analysis and implementation of a prototype}
\vspace{0.5em}\\
\noindent In this chapter, a prototype, who automatically can harvest data and
present it in a way, that it can be used for social engineering attacks, is
developed.

\noindent {\textbf{Chapter \ref*{chapter:evaluation}}: Evaluation}
\vspace{0.5em}\\
\noindent This chapter wants to examine, whether the already described attacks
can be achieved by using the prototype and therefore shows sample attacks. It
will analyse the attack and evaluate the risk against a company or an
individual.

\noindent {\textbf{Chapter \ref*{chapter:conclusion}}: Conclusion}
\vspace{0.5em}\\
\noindent The thesis finally concludes draws together the main findings of the
study.

\newpage

\section{Glossary}
\begin{multicols}{2}

\begin{description}
\item[foo] bar
\end{description}

\end{multicols}
