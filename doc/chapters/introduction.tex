\chapter{Introduction}
\label{chap:introduction}

\section{Problem and Motivation}

Social Networks, such as Facebook\footnote{\url{http://www.facebook.com}},
MySpace\footnote{\url{http://www.myspace.com}} or
Twitter\footnote{\url{http://www.twitter.com}} gained millions of members on
their platforms in the last years. They are widely used in both private and
business networking. Also, they allow individuals to present their own user
profile and to share and keep relationships to other members of those networks.
There is no doubt that social networks are a useful instrument and pose a
communication platform, which many people are using and even more will be using
in the near future~\cite{whitworth2009}.

The nature of social networks is about sharing data, but also often, users are
also presenting their information to an unknown number of strangers. This does
not only raise the question how useful a social network is for a single user,
but also if there are any drawbacks or risks in any form for the single user or
company. Users often put much and frequently sensitive data into their
profiles, as for example~\cite{brown2008} shows. This data however is stored
centrally at the company which provides the network and so the user looses the
control over his data~\cite{fraunhofer2008}.

Of course there is a relation between the data, which the user entered, and the
user himself. Following that thought, another person is not only able to
extract the data but also to obtain new information or rather interpretations,
which were not entered by the user.

Many studies, like~\cite{fraunhofer2008,gross2005} show already, that this can
be a massive intervention into one's privacy. This work however wants to
examine whether such information are a security risk and can be used for
attacks against a company or an individual. More precisely, it wants to assay
how social engineering attacks can be driven against users of social networks,
based on data, which is legally accessible on social networks.

Of course, a distinction between employees of a certain company and private
users has to be made. For example, certain information can be harmless to an
individual, but very well a danger to a company~\cite{mitnick2003}. For example
a phone number of a private person is probably harmless, while the phone number
of a certain office inside a company could be used for a social engineering
attack. In both cases, however, a social engineer saves additional time and
effort to get a specific information about his victim, if he can find the
information needed on a social network. Having to collect the
information by hand, does not only require extra work, but it also includes the
danger of being exposed. A social engineer has to remain undiscovered during
the attack, as he otherwise would not be able to carry out an attack
\cite{mitnick2003}.

Especially big corporations spend hundreds of thousands of dollars for the
security of their IT-infrastructure. An attack on the IT-layer would therefore
be often very laborious. Therefore it is much simpler to bypass the security
mechanisms through social engineering, as it is very cheap and does not require
any superior technology~\cite{winkler1995}. Commonly, a skilled social
engineering does not require anything more than a telephone for such attacks
\cite{mitnick2003}.

Concretely, this study wants to answer the following questioning:
How can data of individuals or companies automatically be extracted
from social networks and presented in a way, that it can be used for a social
engineering attack. In addition, countermeasures against automatic extraction
and social engineering attacks are going to be developed.

\section{Methodology}

In this work, a five-step approach was used to characterize useful information,
extraction and countermeasures. The target was to define the steps needed for
executing an attack like already described.

\begin{enumerate}

\item Data was obtained from the chosen social network from several users. The
data was taken from famous persons on the network, as well as less famous.
Furthermore users, who were very active as well as less active users were
watched.

\item The data gathered in the first step was studied and it was determined,
which attacks are possible. 

\item Three sample attacks, which happened for real in the last years, were
chosen. All of these attacks are described by Mitnick and Simon
\cite{mitnick2003}. The chosen attacks were then transfered to the prototype
and the chosen social network.

\item The attacks were tested on sample profiles for their feasibility and
efficiency

\item Countermeasures were elaborated to mitigate the risk of such attacks.

\end{enumerate}

\newpage
\section{Outline of the thesis}

\let\oldchapterautorefname \chapterautorefname
\def\chapterautorefname{Chapter}

\noindent{\textbf{\autoref*{chap:introduction}}: \nameref{chap:introduction}}
\vspace{0.5em}\\
\noindent This chapter presents an overview of the thesis and introduces the
reader to the topic of new threats of social networks by exploiting social
networks.

\noindent {\textbf{\autoref*{chap:relatedwork}}: \nameref{chap:relatedwork}}
\vspace{0.5em}\\
\noindent The section will outline related works in the field of social
engineering, the attacks and countermeasures. It will present common attacks
and how companies and individuals can protect themselves against such attacks.

\noindent {\textbf{\autoref*{chap:attacks}}: \nameref{chap:attacks}}
\vspace{0.5em}\\
\noindent Three social engineering attacks, which happened for real, are chosen
and being presented to the reader. They are analyzed in order to let the
prototype repeat the attacks.

\noindent {\textbf{\autoref*{chap:analysis}}: \nameref{chap:analysis}}
\vspace{0.5em}\\
\noindent The study takes a deeper look at the social network \Twitter. It will
show the drawbacks and risks of such social networks, the data, which can be
extracted automatically, how sample attacks could look like and finally the
countermeasures against such attacks.

\noindent {\textbf{\autoref*{chap:prototype}}: \nameref{chap:prototype}}
\vspace{0.5em}\\
\noindent In this chapter, a prototype, who automatically can harvest data and
present it in a way, that it can be used for social engineering attacks, is
developed.

\noindent {\textbf{\autoref*{chap:evaluation}}: \nameref{chap:evaluation}}
\vspace{0.5em}\\
\noindent This chapter wants to examine, whether the already described attacks
can be achieved by using the prototype and therefore shows sample attacks. It
will analyse the attack and evaluate the risk against a company or an
individual.

\noindent {\textbf{\autoref*{chap:conclusion}}: \nameref{chap:conclusion}}
\vspace{0.5em}\\
\noindent The thesis finally concludes draws together the main findings of the
study.

\def\chapterautorefname{\oldchapterautorefname}
\newpage

\section{Glossary}
\begin{multicols}{2}

\begin{acronym}[XXXX]

\acro{API}{An application programming interface (API) is an interface in computer science, that defines ways to interact with services from libraries, the operating system or the web.}
\acro{Atom}{The Atom Syndication Format is a format used for web feeds based on the XML Language.}
\acro{GET}{A method for requesting a specified resource using the HTTP protocol.}
\acro{JSON}{JavaScript Object Notation (JSON) is a computer data interchange format which uses the JavaScript syntax.}
\acro{POST}{A method for transferring data to an identified resource using the HTTP protocol.}
\acro{REST}{Representational State Transfer (REST) is style for software architecture for distributed systems. The web is built on that style, for example.}
\acro{RSS}{Really Simple Syndication is a feed format to publish web feeds, like blogs, news, audio or video}
\acro{URL}{A Uniform Resource Locator (URL) specifies how and where an identified resource is available and the methods retrieving it.}
\acro{XML}{Extensible Markup Language (XML) is a general-purpose specification for creating custom markup languages.}

\end{acronym}

\end{multicols}
