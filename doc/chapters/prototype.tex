\chapter{Design, analysis and implementation of a prototype}
\label{chapter:prototype}

\section{Goal}

The goal is to develop a prototype, which can cull and evaluate data of one or
more users of the \Twitter{} social network. The produced data should then be
assigned to real persons, in order to create a fact sheet for a social
engineer. It should be shown, that it is easy to file employees of a certain
company or private persons, just by using legally available data, put online by
the individuals themselves. The filed data should then enable a social
engineer to do social engineering attacks against the weakest point.

The prototype developed for this work should make it possible to find persons
either by their username or by their real name.
It then should evaluate and display every possible information, which can be
gathered legally. The data should be saved locally, which the possibility to
update it.

\section{Description and blueprint of the prototype using software engineering methods}
\comment{better title needed}

\begin{figure}
  \begin{center}
    \includegraphics[width=\textwidth]{prototype_class.1}
    \caption{UML class diagram  of the developed prototype}
  \end{center}
\end{figure}


\section{Used programming languages, tools and libraries}

For the prototype, the \textit{Python}\footnote{http://www.python.org/} programming
language was used. It is a dynamic object-oriented programming language, that
offers strong support with many tools and comes with a large amount of standard
libraries. \textit{Python} allows rapid prototyping, and has many libraries built
in, used for example to call the \Twitter{} API.

The plotting was done using the
\textit{matplotlib}\footnote{http://matplotlib.sourceforge.net/} library, which
is a 2D plotting library, which produces production quality figures and is easy
accessible through the \textit{Python} programming language.

\section{Documentation and explanation of relevant passages in the source code of the prototype}

This section will now outline a few relevant passages of the prototype.
