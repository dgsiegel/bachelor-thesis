\chapter{Design, Analysis and Implementation of a Prototype}
\label{chap:prototype}

This chapter is going to introduce the design, analysis and implementation of
the prototype, which implements the analysis of the previous chapter.

\section{Design Goal}

The goal is to develop a prototype, which can cull and evaluate data of one or
more users of the \Twitter{} social network. The produced data should then be
assigned to real people, in order to create a fact sheet for a social engineer.
It should be shown that it is easy to file employees of a certain company or a
private person, just by using legally available data, put online by the
individuals themselves. The filed data should then enable a social engineer to
do social engineering attacks against the weakest point. Furthermore, the
attacks, which are analyzed in chapter \ref{chap:attacks}, can be repeated by
using the prototype.

The prototype developed for this work allows to find individuals either by
their username or by their real name. Then it evaluates and displays every
possible information, which can be gathered legally, based on the analysis in
section \ref{ssec:ontology}. The data is saved locally, with the possibility to
update it.

\section{Used Programming Languages, Tools and Libraries}

For the prototype, the \textit{Python}\footnote{\url{http://www.python.org/}}
programming language was used. It is a dynamic object-oriented programming
language, that offers strong support with many tools and comes with a large
amount of standard libraries. \textit{Python} allows rapid prototyping, and has
built in many libraries, used for example to call the \Twitter{} API.

The plotting was done using the 2D plotting library
\textit{matplotlib}\footnote{\url{http://matplotlib.sourceforge.net/}}, which
produces production quality figures and is easy accessible through the
\textit{Python} programming language.

\section{Design and Implementation of the Prototype}

The prototype was developed in a modular approach, to allow easy extensibility.
As the \Twitter{} API is in constant change, this helps to keep the prototype
working and allows to extend it for new data sources.

Each information source is developed as a plugin. The plugin downloads the
required data, parses it and prepares an output, from which a fact sheet gets
generated.

The structure of the prototype is outlined in figure \ref{fig:prototype_class}.
The starting point is the \texttt{Prototype} module, which holds the starting
values, loads the plugins, runs them and puts together the output. To allow
further searches for people, the main module holds several options. The
prototype currently supports the following:

\begin{itemize}
  \item Search and analyze a user. If the username passed is not found, a
  search is launched.
  \item Search users by using specific keywords
  \item Search users in a specific location
  \item Search users in a specific range
\end{itemize}

\begin{figure}[ht]
  \begin{center}
    \includegraphics[width=\textwidth]{prototype_class.1}
    \label{fig:prototype_class}
    \caption{UML class diagram of the developed prototype.}
  \end{center}
\end{figure}

After choosing the option, the API object is created, passing the username and
password. If the username and password is empty, the prototype connects
anonymously to the \Twitter{} API.

\begin{figure}[ht]
  \begin{center}
    \includegraphics[width=0.75\textwidth]{prototype_state.1}
    \label{fig:prototype_state}
    \caption{UML state machine diagram of an \Twitter{} API call.}
  \end{center}
\end{figure}

\lstset{language=python}
\lstinputlistingln{38}{38}{../src/prototype.py}

First, the API object gets created, then it calls the \texttt{\_\_init\_\_()}
method of the \texttt{TwitterAPICall} object. This means that the prototype
has an object of the \texttt{TwitterAPI} object, which then creates a new
\texttt{TwitterAPICall} object each time called. At the same time the
\texttt{FileCache} object is created and passed on, which simply stores the
values downloaded into files.
\lstinputlistingln{152}{156}{../src/twitterapi.py}

When the \texttt{TwitterAPICall} object is created, it fills the default
values with those passed by the \texttt{TwitterAPI} object.
\lstinputlistingln{55}{64}{../src/twitterapi.py}

The \texttt{\_\_getattr\_\_()} method, which is a Python default method, is
called every time an attribute of an object is accessed. The prototype exploits
this behaviour for the \Twitter{} API. For example, if the prototype wants to
call the API function \texttt{/users/show}, it just calls the method
\texttt{twitter.users.show()}. The previous mentioned function then just
removes the non-existing attribute and appends it as an argument to a new
\texttt{TwitterAPICall} object.
\lstinputlistingln{66}{71}{../src/twitterapi.py}

If there is no attribute left, the object function is called, where the
prototype exploits the Python intern function \texttt{\_\_call\_\_} to actually
connect to the \Twitter{} API and download data. The actual downloading is
quite trivial. The method just determines if a POST or GET HTTP request is
needed, puts together the username and password, if given, builds the request
URL. Then, it looks if the query is already cached and if there are API calls
available. If so, it does the actual call to the \Twitter{} API. The result is
then cached by using the \texttt{FileCache} object.

\begin{figure}[htb]
  \begin{center}
    \includegraphics[width=0.75\textwidth]{prototype_activity.1}
    \caption{Activity diagram of a generic plugin.}
    \label{fig:prototype_activity}
  \end{center}
\end{figure}

Next, the plugin system is initialized, which simply imports the plugin files
by using the Python method \texttt{\_\_import\_\_()}. The list
\texttt{PLUGINS\_ENABLED} just holds the plugins, which should get loaded, run
and displayed.
\lstinputlistingln{20}{24}{../src/plugin.py}

All plugins are inheritances or subclasses of the \texttt{Plugin} object, which
just holds the methods \texttt{download()}, \texttt{parse()} and
\texttt{output()}. Those methods have to be implemented by each plugin. Figure
\ref{fig:prototype_activity} shows, how a single plugin works.
\lstinputlistingln{8}{17}{../src/plugin.py}

Now, each plugin has to be addressed somehow. As it was dynamically loaded,
there is no structure, which holds the plugin itself yet. However, the
\texttt{Plugin} objects knows which subclasses or children were loaded. The
method \texttt{find\_plugins()} returns a list of the loaded plugins. If a
plugin was not created yet, it creates the specific plugin object.
\lstinputlistingln{27}{35}{../src/plugin.py}

The prototype can now call the three methods \texttt{download()},
\texttt{parse()} and \texttt{output()} of each plugin. The output gets merged
and written to a file.

If a new plugin should be added, a new plugin object has to be created, which
implements the above mentioned methods. Each API call gets cached and therefore
it is no overhead if several modules need the same data to work with. The API
object just downloads the required data once and then just passes the cached
data.
\lstinputlistingln{1}{22}{../src/plugins/testplugin.py}

