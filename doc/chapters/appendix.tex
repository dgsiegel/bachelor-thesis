\chapter{\Twitter{} profile data}
\label{chapter:twitter_profile_data}

\begin{description}
\item[Name] This fields tells the real name of a user, it is limited to 20
            characters. It is required to enter a name, albeit a bogus name is
            also accepted. The field is required.\\
            \textit{Example: Peter Sample}
\item[Username] The username is presented with this field. A username is needed
                to create an URL, which is of the form \url{http://www.twitter.com/[username]}.
                Therefore only letters, numbers and underscore are allowed. The maximum length
                is set to 15 characters. The field is required.\\
                \textit{Example: petersample}
\item[User ID] The user id which is handed out automatically. The sequence is
               not known, however it is mostly agreed, that the user id is not given out
               sequentially.\\
               \textit{Example: 1234567}
\item[E-mail] Though not visible publicly, a \Twitter{} user has to provide an
              e-mail address. Even if an e-mail address is not provided, it
              can be build out of a naming scheme and the already
              given data, like the username and the real name \cite{brown2008}. If the user
              is an employee of a certain company, this might be even easier,
              as companies often have certain naming schemes. The field is required.\\
              \textit{Example: petersample@example.com}
\item[More Info URL] The URL is used to link a visitor of a profile to further
                     websites, like the blog of the owner. It can link to
                     any site on the internet. The shown component of the
                     URL is 17 characters. Except XSS prevention,
                     \Twitter{} does not rewrite of the URL, and port numbers
                     after the TLD, spaces, German umlauts and UTF-8 characters were
                     accepted.\\
                     \textit{Example:} \url{http://www.petersample.com/},\\
                     \texttt{http://www.peter sample.com:8080/äöü$\Gamma\Lambda\Sigma\Psi$.htm}
\item[One Line Bio] A short sentence can be shown on the profile page, where a
                    user can describe himself. It it limited to 160
                    characters.\\
                    \textit{Example: I am a computer science expert and work
                    for example.com}
\item[Location] The location will also be shown on the profile page and is
                limited to 30 characters.\\
                \textit{Example: Munich}
\item[Picture] A user can insert a picture of himself, although it is not
               required to do so. Moreover, if a user publishes an image, it does not have to
               show himself, but can also be anything else.
\item[Profile Creation date] The date, the profile was created.\\
              \textit{Example: Fri Nov 02 00:17:11 +0000 2007}
\item[Following] Describes a list of users, the profile owner is \textit{following}.
                 To see the list, one has to be logged in.
\item[Followers] This gives us a list of users on the \Twitter{} network, who
                 are \textit{following} the profile owner. To see the list, one has to be logged in.
\item[Friends] This are users, who have a bidirectional connection. That means,
               that user A is following user B and vice versa.
\item[Favorites] A user can mark his messages or the messages from other users as
                 a favorite, which displays a
                 yellow star beneath them. A users favorite messages can be viewed
                 by visiting \url{http://twitter.com/favorites?user=[username]}.
\item[Messages] Finally the messages, which are composed by the actual message,
                which is limited to 140 characters, the time, when the message
                was written and by which mean. A message is identified by a unique
                id, e.g. 1767572233 and can be accessed by
                \url{http://twitter.com/[username]/status/[message id]}.

                \Twitter{} also offers special commands, which can enhance a
                message or turn a \Twitter{} into a query tool.
  \begin{description}
    \item[@username + message]
      A message can be sent directly to another person, though still visible
      publicly.\\
      \textit{Example: @petersample yes, you are totally right!}

    \item[D username + message]
      In contrast to the command above, this message is entirely private and
      not visible by other users.\\
      \textit{Example: D petersample this is a private message for peter!}

    \item[WHOIS username]
      retrieves information of any public user on the \Twitter{} platform.
      Currently, this command returns the real name, the date since the user
      has an account on \Twitter{}, the one line bio, the website and the
      location.\\
      \textit{Example: WHOIS petersample}\\
      \textit{Answer: Peter Sample, since May 2009. bio: I am a computer
              science expert and work for example.com. location: Munich web:
              \url{http://www.petersample.com/}}

    \item[GET username]
      gets the last message this user posted.\\
      \textit{Example: GET petersample}\\
      \textit{Answer: petersample: This is my latest message (1 day ago)}

    \item[NUDGE username]
      sends a note to the user, reminding him to post a message.\\
      \textit{Example: NUDGE petersample}\\
      \textit{Answer (sent to petersample): You've been nudged! [my-username] wants to know what you're doing. Reply to this
      message to update your Twitter friends.}

    \item[FAV username]
      marks the last message of username as a favorite\\
      \textit{Example: FAV petersample}

    \item[STATS]
      returns the number of followers and how many users the account itself is
      following.\\
      \textit{Example: STATS}\\
      \textit{Answer: followers: 135 following: 47}

    \item[INVITE phone number]
      sends an invite SMS to the phone number\\
      \textit{Example: INVITE 123 456 7890}

\end{description}
\end{description}

%\chapter{Source code}
%\section{Detailed Validation Results}
%\label{chapter:sourcecode}
%
%Here comes the source code
