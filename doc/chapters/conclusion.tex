\chapter{Conclusion}
\label{chap:conclusion}

The work began with the problem and motivation of this topic and defined a
methodology. It then tried to define social engineering, the attacks and
countermeasures. Also, common attacks were presented and countermeasures
against them. Next, three social engineering attacks were analyzed in-depth, in
order to apply them to the chosen social network afterwards. Then, the
\Twitter{} social network was dissected, while keeping the main focus on the
risks this social network exposes and the relevant data, which could be used
for social engineering attacks. It also tried to come up with countermeasures
against the automatic extraction of sensitive data. Having the requirements and
the analysis done, the prototype was implemented. The prototype was developed
to quickly adapt itself on the fast changing \Twitter{} API, but also on new
information sources or different representations. Last, the prototype was used
to drive three sample social engineering attacks. The result was then analyzed
and evaluated.

The evaluation showed, that using the prototype would make the life of a social
engineer much easier, as in most cases, he can use the prototype for doing the
research about his victim. This also dwindles the chance of being detected,
which could baffle the social engineering attack.

The prototype was kept very extensible, in order to easily attach new
information sources or different representation of data. Furthermore it has to
be said, that the possibilities of text analysis, especially on a social
network like \Twitter{} is not for a long time bailed out yet. Spending more
effort on that, would result into a even more detailed fact sheet, usable for
even more social engineering attacks and especially for more detailed planned
attacks.

It could also be extended to support more networks, or even using some search
engines, in order to gather even more data about several persons. This also,
would result in a more detailed achievement.

One scenario, that will probably be more present in the next years, could be an
automatic extraction and attacking framework. As this work already showed, how
to automatically extract data out of a social network, it is only a matter of
time, until somebody will use that data for together with predefined social
engineering attacks. The example of the automated phishing mail was mentioned
before in this work.

While social networks become more famous and attractive every day, the security
of such networks does not have to be left aside. Much work is still needed in
this area, until people can use those networks safely, without having the fear
of becoming a victim of a social engineering attack.
