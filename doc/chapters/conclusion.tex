\chapter{Conclusion}
\label{chap:conclusion}

The work began with the problem and motivation of exploiting social networks
using social engineering. It defined the main goal of the work, which is to
give an answer to how data of individuals or companies can be automatically
extracted from social networks and presented in a way, that it can be used for
a social engineering attack. Furthermore, countermeasures against automatic
extraction and social engineering attacks were analyzed and developed.

A methodology of how to characterize useful information, extraction and
countermeasures was defined. This determined the further analysis and the
constitution of the work.

It then tried to define social engineering, the attacks and countermeasures.
Therefore the social engineering life cycle was introduced, which outlines a
social engineering attack and is constituted of four phases: research, developing
rapport and trust, exploiting trust and utilizing information. A single cycle
of course is not limited to a single cycle, but can contain several other
cycles until the objective is reached. The research phase consists of several
methodologies to gain as much information possible about the victim, which then
can be used to develop rapport and trust in the next phase. A social engineer
will make contact to the victim using the information obtained in the phase
before. This trust, which was established is now exploited to procure the
information needed, which then is used in the last phase. Also, the threats
and risks created through social engineering were described. To obviate social
engineering attacks, countermeasures were demonstrated, however there is no way
to be absolutely shielded against such attacks. While data retrieval of social
networks usually does not count to the methods of the research phase of a
social engineering attack, it was showed, how social networks can replace or
enhance several \textit{classic} methods. Furthermore, not just the data
retrieval, which was described as \textit{passive attacks}, was analyzed, but
also attacks, which are exploiting a social network directly and are known as
\textit{active attacks}.

Next, three social engineering attacks were analyzed in-depth, in
order to apply them to the chosen social network afterwards. The attacks
themselves were analyzed, but also the information needed, and how they could
be applied to a social network. The first attack is aimed towards a private
person and should represent this group of attacks. The other two are aimed
against companies, both using data, which often is not easy available and has
to be constructed by the social engineer. The last one, however requires
sensitive data, which is not supposed to be put online.

Then, the \Twitter{} social network was chosen as the social network, the
attacks should be applied on. It was dissected, while keeping the main focus on the
risks this social network exposes and the relevant data, which could be used
for social engineering attacks. It was determined, how automatic extraction of
data could actually work and the danger of being detected, which is practical
not existent. Further the data, which could be extracted was classified in
several groups, which then could be used for the attacks. Again the threats and
risks of this concrete information retrieval was discussed and the
countermeasures against automatic extraction were illustrated. However, there
is no way to exclude the threat and risk. All countermeasures however are able
to decrease the chance of being a victim of a social engineering attack based
on data of a social network.

Having the requirements and the analysis done, the prototype, which can
automatically extract data from the \Twitter{} social network was designed and
implemented. The prototype was developed to quickly adapt itself on the fast
changing \Twitter{} API, but also on new information sources or different
representations. It is able to extract data from the social network and present
it in a \textit{fact sheet}, which then can help a social engineer to replace
the research phase by just using the prototype and not exposing himself to the
victim.

Last, the prototype was used to drive three sample social engineering attacks.
The attack against a private person was successful and showed, that similar
attacks are possible too. The second attack was successful too, however
including the possibility of having to run another social engineering attack to
gain specific data. Though, this attack can be described as successful too, as
most of the data required was obtained and a second attack is not always
needed. Only the last attack failed, which required sensitive information,
which should not be possible to find on a social network. But of course, the
prototype managed to retrieve most of the data about the company.

The evaluation showed, that using the prototype would make the life of a social
engineer much easier, as in most cases, he can use the prototype for doing the
research about his victim. This also dwindles the chance of being detected,
which could baffle the social engineering attack.

As the attacks driven before already evidence, both targeted groups are
vulnerable to such attacks. Regarding to private persons, there is a wide range
of attacks possible sprawling from online attacks, such as phishing or identity
theft, to even \textit{offline} attacks. Most social engineering attacks, which
involve the knowledge of a certain piece of information, seem to be possible by
exploiting the information found a social network. And even delicts, which do
not involve social engineering seem to be possible. For example as an extreme
case not related to social engineering, even burglary would be possible, if a
picklock has the information, that the victim has holidays and nobody is in his
home.

Concerning companies as the target of the described attacks, the companies
themselves are very vulnerable as well, as the work showed. The company
structure as well as sensitive data could be extracted from the social network.
Combining the automatic extraction with a social engineer, who is able to gain
even more information and to establish a relationship to several employees is a
very dangerous instrument. It was showed, that especially large companies were
already extremely insecure. Having a method for the automatic extraction of
company related information, this just increases the risk of an social
engineering attack.

Combining both groups, could result in a devastating result for the victim. An
attacker may drive some attacks against several private persons, who are related to a
specific company. Then he could exploit them to get access at the company, to
execute his attack. This is probably the worst case scenario for a company.

The prototype was kept very extensible, in order to easily attach new
information sources or different representation of data. Furthermore it has to
be said, that the possibilities of text analysis, especially on a social
network like \Twitter{} is not for a long time bailed out yet. Spending more
effort on that, would result into a even more detailed fact sheet, usable for
even more social engineering attacks and especially for more detailed planned
attacks.

It could also be extended to support more networks, or even using some search
engines, in order to gather even more data about several persons. This also,
would result in a more detailed achievement.

One scenario, that will probably be more present in the next years, could be an
automatic extraction and attacking framework. As this work already showed, how
to automatically extract data out of a social network, it is only a matter of
time, until somebody will use that data for together with predefined social
engineering attacks. The example of the automated phishing mail was mentioned
before in this work.

While social networks become more famous and attractive every day, the security
of such networks does not have to be left aside. Much work is still needed in
this area, until people can use those networks safely, without having the fear
of becoming a victim of a social engineering attack.
