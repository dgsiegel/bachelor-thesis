\chapter{Analysis of the social network \Twitter}
\label{chapter:analysis}

The work now wants to take a closer look at a concrete social network.
There are of course many social networks, which can be used for harvesting
data, however, as a case study, the author wants to concentrate on a big social
network, which is widely used by employees and individuals. Moreover,
an \textit{API} is needed to gather data automatically.

As a case study, the author had the choice between \textit{Facebook} and
\Twitter. Both are widely used, however \Twitter{} was chosen due to many
already existing programs and platform bindings. Furthermore \Twitter{} is
somewhat more dynamic than \textit{Facebook}, as it offers the possibility to
parse \Twitter{} messages and not just static data.

\Twitter{} is a popular social networking and micro-blogging service, that
enables users to post messages and to let other users follow those messages.
The term \textit{micro-blogging} describes a form of communication, that
consists of brief messages in text form, which then can be send over a variety
of ways, like instant messages, mobile phones, e-mail or other \cite{java2007}.

Micro-blogging itself is relatively new, though already widely used and
provided by services like
\textit{Twitter}\footnote{http://www.twitter.com},
\textit{identi.ca}\footnote{http://identi.ca},
\textit{Jaiku}\footnote{http://www.jaiku.com} and others. \Twitter{} is one of
the most popular micro-blogging services \cite{java2007} and currently .

\begin{figure}
\begin{center}
\begin{tikzpicture}[scale=0.75, transform shape,
                    root concept/.append style={concept color=skyblue1},
                    level 1 concept/.append style={concept color=chameleon2},
                    text=white, mindmap]

\tikzstyle{every annotation}=[fill=skyblue3, font=\sf]

  \node[concept] (Twitter) {Twitter}
    child[concept, grow=160] {node [concept] {Web Interface}}
    child[concept, grow=125] {node [concept] {Twitter API}
    child[concept, grow=left] {node [concept] {User Applications}}}
    child[concept, grow=90] {node [concept] {Facebook}}
    child[concept, grow=55] {node [concept] (top) {IM}}
    child[concept, grow=20] {node [concept] {SMS}}
%
    child[concept, grow=200] {node [concept] {Web Interface}}
    child[concept, grow=235] {node [concept] {Twitter API}
    child[concept, grow=170] {node [concept] {RSS}}
    child[concept, grow=210] {node [concept] {User Applications}}}
    child[concept, grow=270] {node [concept] {Facebook}}
    child[concept, grow=305] {node [concept] (bottom) {SMS}}
    child[concept, grow=340] {node [concept] {IM}};

  \node[annotation, above] at (top.north east) {Twitter Input Methods};
  \node[annotation, below] at (bottom.south east) {Twitter Output Methods};

  \draw (2,0) -- (6,0);
  \draw (-2,0) -- (-6,0);

\end{tikzpicture}
\caption{\Twitter{} input and output methods, on the basis of \cite{krishnamurthy2008}}
\end{center}
\end{figure}



\section{Reasons for choosing this social network}

\section{Threats and risks}

\section{Relevant data and the security risks of those at individuals and companies}
\subsection{The challenge of extracting data automatically}
\subsection{Ontology and classification of the data}
\subsection{Automatic extraction and evaluation}

\section{Accomplishment of attacks}

\section{Countermeasures}
