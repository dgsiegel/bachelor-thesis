\chapter{Introduction}
\label{chapter:introduction}

\section{Problem and Motivation}

Social Networks, such as Facebook\footnote{\url{http://www.facebook.com}},
MySpace\footnote{\url{http://www.myspace.com}} or
Twitter\footnote{\url{http://www.twitter.com}} gained million of members on
their platforms in the last years. They are widely used in both private and
business networking. Services, like the one just mentioned, allow individuals
to present their own identity and to share and keep relationships to other
members of those services. Often, they are also presenting their information to
an unknown number of strangers.


This brings us not only to the question how useful a social network is for a
single user, but also if there are any drawbacks or risks.  As for example
\cite{brown2008} shows, users often put much and frequently sensitive data into
their profiles. This data however is stored centrally at the company which
provides the network and so, often the user looses the control over his
data~\cite{fraunhofer2008}.

There is of course a relation between the data the user entered and the user
himself. Following that thought, another person is not only able to extract the
data but also to obtain new information or rather interpretations, which were
not entered by the user.

Many studies, like \cite{fraunhofer2008,gross2005} show already, that this can
be a massive intervention into one's privacy. But this study wants to examine
whether such information are a security risk.

Of course, a distinction between employees of a certain company and private
users has to be made. For example, certain information can be harmless to an
individual, but very well a danger to a company \cite{mitnick2003}. In both
cases, however, a social engineer saves an additional step, for example a
telephone call or similar, to get that specific information. That step does not
only require extra work, but also the danger of being exposed. Without his
anonymity, a social engineer could not carry out an attack \cite{mitnick2003}.

Especially in big corporations spend hundreds of thousands or dollars for the
security of their IT-infrastructure. An attack at the IT-layer would therefore
often be very laborious. Therefore it is much simpler bypass the security
mechanisms through social engineering, as it very cheap and does not require
any superior technology \cite{winkler1995}. Commonly, a skilled social
engineering does not require anything more than a telephone for such attacks
\cite{mitnick2003}.


Sind nun aber diese Informationen wirklich so sicherheitsrelevant, dass sie ein
Risiko darstellen können? Ja, denn für Angreifer sind dies nun optimale
Voraussetzungen, um Angriffe gegen bestimmte Personen oder Unternehmen zu
fahren, wenn bereits viele sensible Daten sehr einfach über oben genannte
Plattformen schnell, einfach und kostengünstig extrahiert werden können [4]. So
kann bereits die Information, dass ein Mitarbeiter einer Firma im Urlaub oder
über die Mittagszeit abwesend ist, dazu ausgenutzt werden, sich als diese
Person auszugeben und so einen Identitätsdiebstahl auszuüben. Eine sehr häufig
benutzte Technik des Social Engineering [6].

Doch wie kann sich nun eine Privatperson oder ein Unternehmen und dessen
Angestellte gegen solche Angriffe schützen? Hier sollen Vorschläge erarbeitet
werden, die als Gegenmaßnahmen benutzt werden können um so Social Engineering
Angriffe einzudämmen bzw. auszuschließen.

Welche Informationen nun aber für Social Engineering Angriffe in Betracht
gezogen werden können, welches Sicherheitsrisiko diese darstellen und wie man
sich davor schützen kann, soll der inhaltliche Sinn dieser Bachelor Thesis
sein.

Die konkrete Fragestellung lautet: \glqq{}Wie können Daten bestimmter
Individuen bzw. Unternehmen über Soziale Plattformen automatisch extrahiert
und so dargestellt werden, dass sie für einen Social Engineering Angriff
benutzt werden können. Zudem sollen Gegenmaßnahmen zu diesen Angriffen
erarbeitet werden.\grqq{}


\section{Threats and risks through social engineering attacks}
\section{Outline of the thesis}
%%--------------------------------------------------------------------
%\section*{Part I: Introduction and Theory}
%%
%\noindent {\scshape Chapter 1: Introduction}  \vspace{1mm}
%%
%\noindent  This chapter presents an overview of the thesis and it purpose.
%Furthermore, it will discuss the sense of life in a very general approach.  \\
%%
%\noindent {\scshape Chapter 2: Theory}  \vspace{1mm}
%%
%\noindent  No thesis without theory.   \\
%%
%%--------------------------------------------------------------------
%\section*{Part II: The Real Work}
%%
%\noindent {\scshape Chapter 3: Overview}  \vspace{1mm}
%%
%\noindent  This chapter presents the requirements for the process.
%~            
%
\section{Glossary}
